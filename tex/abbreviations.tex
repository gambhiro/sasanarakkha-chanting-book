\chapter{Abbreviations}
\label{abbreviations}

\ifafiveversion
\begin{tabular}{@{}ll@{}}
  \anglebracketleft\ \hspace{-0.5mm}... \hspace{-0.85mm}\anglebracketright\ & \hspace{1.30mm}Only recited by the leader \\
  \hspace{0.1cm} \abbrbreathmark\ & \hspace{1.45mm}Breathing pause \\
\end{tabular}
\fi

\ifasixversion
\begin{tabular}{@{}ll@{}}
  \anglebracketleft\ \hspace{-0.5mm}... \hspace{-0.70mm}\anglebracketright\ & \hspace{0.90mm}Only recited by the leader \\
  \hspace{0.1cm} \abbrbreathmark\ & \hspace{0.90mm}Breathing pause \\
\end{tabular}
\fi

\ifbsixversion
\begin{tabular}{@{}ll@{}}
  \anglebracketleft\ \hspace{-0.5mm}... \hspace{-0.85mm}\anglebracketright\ & \hspace{1.40mm}Only recited by the leader \\
  \hspace{0.1cm} \abbrbreathmark\ & \hspace{1.60mm}Breathing pause \\
\end{tabular}
\fi

\begin{tabular}{@{}llll@{}}
  DN    & Dīgha Nikāya                                        \\
  MN    & Majjhima Nikāya                                     \\
  SN    & Saṁyutta Nikāya                                     \\
  AN    & Aṅguttara Nikāya                                    \\
  Khp   & Khuddakapāṭha                                       \\
  Dhp   & Dhammapada                                          \\
  Ud    & Udāna                                               \\
  Snp   & Sutta Nipāta                                        \\
  Thag  & Theragāthā                                          \\
  Ja    & Jātaka                                              \\
  Pv    & Petavatthu                                          \\
  Vibh  & Abhidhamma Vibhaṅga                                 \\
  Dhs   & Dhammasaṅganī                                       \\
\end{tabular}

\begin{tabular}{@{}llll@{}}
  A     & Aṭṭhakathā (Commentary)                             \\
  MJG   & Mahā-jaya-maṅgala-gāthā (Sri Lanka)                 \\
  Trad  & Traditional verses not found in the original Pāli   \\
  WPN   & Wat Pah Nanachat Buddhist Chanting (2014)           \\
\end{tabular}

\medskip

Wisdom Publication sources: Nikāya and sutta \# (eg. DN 1)\\
P.T.S. sources: Nikāya, volume \#, page \# (eg. D i 1)
