\section{Chanting Leader Guidelines}

  It is both a great honour and also a great responsibility to lead a group of fellow Dhamma practitioners through Dhamma recitations. These guidelines provide tips that can help facilitate a smooth and enjoyable experience for the group of participants, as well as for the chanting leader himself.

The following points should be considered:

\ifafiveversion
\begin{itemize}
  \item Accurate Pāli Pronunciation
  \item Sufficient Audibility and Volume
  \item Determining a Suitable Tempo
  \item Tempo Consistency
  \item Determining a Suitable Pitch
  \item Pitch Consistency
  \item Continuity
\end{itemize}
\fi

\ifasixversion
\begin{packeditemize}
  \item Accurate Pāli Pronunciation
  \item Sufficient Audibility and Volume
  \item Determining a Suitable Tempo
  \item Tempo Consistency
  \item Determining a Suitable Pitch
  \item Pitch Consistency
  \item Continuity
\end{packeditemize}
\fi

\ifbsixversion
\begin{packeditemize}
  \item Accurate Pāli Pronunciation
  \item Sufficient Audibility and Volume
  \item Determining a Suitable Tempo
  \item Tempo Consistency
  \item Determining a Suitable Pitch
  \item Pitch Consistency
  \item Continuity
\end{packeditemize}
\fi

\ifbsixversion\vspace{-0.1cm}\fi

\subsection*{Accurate Pāli Pronunciation}

  Within the domain of Pāli pronunciation, the chanting leader shall pay special attention to long and short syllables, since this would otherwise result in a asynchronous recitation with those who accurately pronounce long and short syllables. To obtain good Pāli pronunciation, frequent chanting with those who have good pronunciation or listening and chanting along with audio recordings of well-pronounced Pāli recitations can go a long way. This helps to identify and even out idiosyncrasies that may have arisen. Of course, all other elements of accurate Pāli pronunciation too should be adhered to as much as possible (voiceless vs. voiced, aspirate vs. non-aspirate consonants etc.), but the difference between long and short syllables is the most important one in the present context.

\subsection*{Sufficient Audibility and Volume}

  In order for others to be able to follow the chanting style of the chanting leader accurately, it is tantamount that he is clearly audible. The voice of the chanting leader should be louder than that of the loudest group member, so that others can follow the chanting leader's lead with ease, rather than having to make an effort to find his voice. This is not to say that the chanting leader needs to blast everyone to such a degree that only his voice can be heard. This would defeat the purpose of group recitation. As often in Dhamma practice, a healthy balance between extremes needs to be found. However, when in doubt it is preferable if the chanting leader errs on the side of ``too loud'', rather than ``too soft''. Both newcomers as well as those who have memorized the chants, but are not yet fully proficient in their recollection of the chants, benefit greatly from a chanting leader whose voice has a decent volume and is clearly audible. While some individuals naturally have a voice that resonates strongly, or have trained their voice to that extent, those who are not so disposed may use a microphone for support. Using a microphone also allows the chanting leader to recite in a more relaxed manner, rather than having to constantly make sure that his voice carries to all corners of the room or hall. For chanting periods that take more than 20 minutes, there is a tendency for one's voice to become softer and thus less penetrating compared to how it was at the beginning of the chanting session. Again, a microphone or increased effort can help to remedy this tendency.

\subsection*{Determining a Suitable Tempo}

Everyone has a preferred tempo at which one finds it pleasant to recite. It is helpful for the chanting leader to choose a tempo that allows him to maintain a clear pronunciation of Pāli terms. Even if one is able to chant accurately at high speed, it would be very difficult for newcomers and those who are not familiar with Pāli to follow along. Pāli is not our mother tongue, and fluent reading requires a lot of skill and practice. It is, therefore, important when chanting with others to choose a tempo that they can follow along comfortably without causing them to feel stressed by the end of the chant, as if having participated in a rap battle. Choosing a more meditative pace also allows the mind to process the Dhamma while reciting it; a task that is not easy for beginners. Slowing down aids in allowing for reflections and insights to arise during Dhamma recitation. At the same time, if the pace is too slow, some individuals may feel bored, or even fall asleep. Again, the right tempo usually lies somewhere between the extremes. Private recitation in one's kuṭi or walking path, undertaken for the sake of maintaining chants in memory by rehearsing them from time to time, is usually done much faster than how one would recite with a group. This allows one to rehearse large amounts of texts in a short time, which is useful when wanting to maintain a large body of chants in one's memory. In contrast, group recitation should follow a more moderate pace to fulfill the earlier mentioned objectives.

A regular chanting leader may ask for feedback from others regarding whether his chanting tempo is comfortable to chant along with or whether it is too fast or slow. The experience of newcomers should receive special consideration.

\ifbsixversion\vspace{-0.14cm}\fi

\subsection*{Tempo Consistency}

  Once a suitable tempo has been chosen to start one's recitation, it may be challenging to maintain the chosen tempo throughout the particular chant, or even throughout the entire chanting session. For most individuals there is a natural tendency to speed up as time goes by. It is the chanting leader's task to prevent that from happening, unless he finds out that the speed he has chosen at the beginning of the chant is too fast or slow for himself and/or others. The ideal time for tempo adjustments is between two chants, not during a chant. The introduction line, e.g. ``Handa mayaṁ...'', or ``Evaṁ me sutaṁ'', can be used as an indicator for how fast or slow the chanting leader wants to recite, whereas the rest of the group shall follow whatever tempo the chanting leader has chosen. If another group member, especially one with a strong and predominant voice has a tendency to change tempo midway, or lengthen/shorten/skip syllables due to inaccurate Pāli pronunciation, the chanting leader must not follow his lead. If this is a recurrent problem with a particular group member, it is worthwhile to talk to him privately and request that he please chant more softly so that other group members can more accurately and easily follow the chanting leader's lead.

\subsection*{Determining a Suitable Pitch}

  Much of what has been stated in preceeding sections on chanting tempo, applies equally to one's chanting pitch. Everyone has his own preferred pitch, at which his voice sounds most natural and requires the least effort to produce sound. It is helpful for the chanting leader to choose a pitch that is not too far from his natural pitch; otherwise it can become very exhausting, especially when recitation continues for longer periods. However, a certain degree of adjustment may be necessary on the part of the chanting leader, if his natural pitch happens to be far outside the average person's range. If it is very high or very low, and thus different from that of some -- or most -- of the chanting participants, it can be difficult for them to match. A regular chanting leader can learn by asking for feedback from others, as to whether they find his pitch comfortable to chant along with, or whether they find it too high or too low. If it turns out that many people point out they find his pitch to be very high or low, he may deliberately adjust his pitch, while still remaining within his own comfort zone. Listening to a reference note before starting to recite, may be helpful for the chanting leader to hit the right pitch every time. A musical ear too can help, but this is very difficult to train; some even claim it's impossible to develop. Someone who aspires to polish his skills in this area should learn one or two tricks for how to find his own comfortable pitch every time he starts to lead a chant, or he can privately approach the Saṅghapariṇāyaka for some guidance.

\subsection*{Pitch Consistency}

  Once a suitable pitch has been chosen to start one's recitation, it can be a challenging task to maintain the chosen pitch throughout one chant, and even more so throughout the entire chanting session. For many individuals there is a natural tendency to gradually drag the pitch lower and lower as time goes by. It is the chanting leader's task to prevent that from happening, unless he feels that the pitch he has chosen at the beginning of the chant is too high or low for himself and/or others. In this case, the ideal time for making pitch corrections, is between two chants. The introduction line, e.g. ``Handa mayaṁ...'', sets the tone, whereby the rest of the group tunes in at whatever pitch the chanting leader has chosen. Once a certain pitch has been selected, one tries to maintain it throughout the chant. If another group member, especially one with a strong and predominant voice, has a tendency to change pitch midway, one should try not to follow his lead but keep chanting in the original pitch. If this is a recurrent problem with a particular group member, it is worthwhile to talk to him privately and request that he chant more softly, so that other group members can more accurately and easily follow the chanting leader's lead.

\subsection*{Continuity}

  Individuals who do not know how to recite a particular text due to moderate prior exposure, as well as those who have memorized a text poorly, rely heavily on the chanting leader. They benefit greatly from a chanting leader whose voice can be heard throughout the entire chant. Continuity can be difficult to maintain when there is a need to breathe at places other than those when there is a breath mark, or when having to cough. At such times a chanting leader can help the group by choosing his pauses skilfully. It is particularly at the beginning of sentences/lines that certain group members may get stuck due to a lapse of memory. It is here when the voice of the chanting leader can be very helpful to get them back on track. Likewise, at the end of long lines/sentences, the chanting leader's voice is important. Some members of the group may have run out of breath midway, and may therefore skip the last few words and start chanting again only at the next line, whereupon one may notice that the overall volume has decreased towards the end of the sentence. Here too it is important that at least the chanting leader continues until the very end of the sentence. The importance of the chanting leader's voice to be present at the beginning and end of the line/sentence, makes the middle of the sentence the ideal time for him to take a short breathing break if necessary. In the middle of sentences, the absence of his voice may barely be noticed, since the rest of the group keeps the chant going. The second-best choice for a short break is the end of the line. However, it is imperative that the chanting leader's voice be clearly present at the beginning of the line. In this way he helps to keep the chant going without interruption, and those who recite by heart can trust and rely on the chanting leader's voice for their support.

\subsection*{Summary}

  The above points and principles are meant to function as a guideline for aspiring chanting leaders to make their own experience as smooth as possible, without having to be stressed or insecure about making mistakes. Moreover, the rest of the group benefits from well-trained chanting leaders, who give their fellow Saṅgha members a sense of ease and comfort while reciting the words of the Buddha, allowing them to reflect on the meaning along the way. After all, recitation of the Dhamma is one of the gateways to liberation (AN 5.209).

  Thanks to all Saṅgha members who are willing to share this responsibility, by taking turns as chanting leaders.
