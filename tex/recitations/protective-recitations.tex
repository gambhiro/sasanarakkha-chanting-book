
\ifdigitalversion
  \chapterOpeningPage{protective-recitations-compressed.jpg}
\else
  \chapterOpeningPage{protective-recitations.jpg}
\fi

\chapter{Protective Recitations}

\clearpage



\sectionSubTitle{Deva-ārādhanā}
\section{Invitation to the Devas}
\label{invitation-to-the-devas}

\ifafiveversion\vspace{-0.5em}\fi
\ifninebythirteenversion\vspace{-0.8em}\fi
\ifbfiveversion\vspace{-0.2em}\fi

\linkdest{endnote124-body}
\begin{leader-pharitvana}
  \anglebracketleft\ \hspace{-0.5mm}Pharitvāna\makeatletter\hyperlink{endnote124-appendix}\Hy@raisedlink{{\pagenote{%
        \hypertarget{endnote124-appendix}{\hyperlink{endnote124-body}{The ``Invitation to the Devas'' is traditionally performed as a solo introduction by the thirdmost senior monk. However, any monk who has been invited by the \textit{Saṅghatthera} can do so.}}}}}\makeatother \thinspace
  mettaṁ samettā bhadantā\\
  Avikkhitta-cittā parittaṁ bhaṇantu\\
  Sagge kāme ca rūpe giri-sikhara-taṭe c'antalikkhe vimāne\\
  Dīpe raṭṭhe ca gāme taru-vana-gahane geha-vatthumhi khette\\
  Bhummā c'āyantu devā jala-thala-visame yakkha-gandhabba-nāgā\\
  Tiṭṭhantā santike yaṁ muni-vara-vacanaṁ sādhavo me suṇantu\\
  Buddha-dassana-kālo ayam'bhadantā\\
  \linkdest{endnote125-body}
  Dhamma-ssavana-kālo ayam'bhadantā\makeatletter\hyperlink{endnote125-appendix}\Hy@raisedlink{{\pagenote{%
        \hypertarget{endnote125-appendix}{\hyperlink{endnote125-body}{When chanted for laypeople, only the second line (\textit{Dhamma-ssavana kālo}...) is recited, which is repeated three times.}}}}}\makeatother\\
  \linkdest{endnote126-body}
  Saṅgha-payirupāsana-kālo ayam'bhadantā\ifdigitalversion\makeatletter\hyperlink{endnote126-appendix}\Hy@raisedlink{{\pagenote{%
        \hypertarget{endnote126-appendix}{\hyperlink{endnote126-body}{WPN: ``\textit{Payirūpāsana}''. \textit{Pari} + \textit{upa} = \textit{payirupa}}}}}}\makeatother\thinspace\fi \hspace{-0.5mm} \anglebracketright\
\end{leader-pharitvana}

\begin{leader-pharitvana-english}
  Kind, venerable sirs: having spread thoughts \ifninebythirteenversion \\ \else \fi of good will,\\
  Recite the protective chant with undistracted mind.\\
  Those in the heavens of sensuality \& form,\\
  On peaks \& mountain precipices, in palaces floating \ifninebythirteenversion \\ \else \fi in the sky,\\
  In islands, countries, \& towns,\\
  In groves of trees \& thickets,\\
  Around homesites \& fields.\\
  And the earth-devas, spirits, heavenly minstrels,\\ \& nāgas who are nearby,\\
  In different waters and lands:\\
  May they come \& listen with approval as I recite\\ the word of the excellent sage.\\
\ifafiveversion\clearpage\fi
\ifbfiveversion\clearpage\fi
  This is the time to see the Buddha, venerable sirs.\\
  This is the time to listen to the Dhamma, venerable sirs.\\
  This is the time to attend to the Saṅgha, venerable sirs.
\end{leader-pharitvana-english}

\suttaRef{[Thai]}



\sectionSubTitle{Pubba-bhāga-nama-kāra-pāṭho}
\section{Passage on Preliminary Paying of Homage}
\label{passage-on-preliminary-paying-of-homage-protective}

\ifninebythirteenversion

\begin{pali-hang-together}
  \anglebracketleft\ \hspace{-0.5mm}Namo tassa \hspace{-0.5mm}\anglebracketright\ bhagavato arahato sammāsambuddhassa
\end{pali-hang-together}

\begin{pali-hang-together}
\hfill{[3x]}
\end{pali-hang-together}

\else

\begin{pali-hang-together}
  \anglebracketleft\ \hspace{-0.5mm}Namo tassa \hspace{-0.5mm}\anglebracketright\ bhagavato arahato\\ sammāsambuddhassa \hfill{[3x]}
\end{pali-hang-together}

\fi

\begin{english}
  Homage to the Blessed, Worthy, and Perfectly Enlightened One
\end{english}

\suttaRef{[DN 21]}



\sectionSubTitle{Saraṇa-gamana-pāṭho}
\section{Passage on Going for Refuge}
\label{passage-on-going-for-refuge}

\begin{pali-hang-together}
  \anglebracketleft\ \hspace{-0.5mm}Buddhaṁ \hspace{-0.5mm}\anglebracketright\  saraṇaṁ gacchāmi\\
  Dhammaṁ saraṇaṁ gacchāmi\\
  Saṅghaṁ saraṇaṁ gacchāmi
\end{pali-hang-together}

\begin{english-verses}
  To the Buddha I go for refuge.\\
  To the Dhamma I go for refuge.\\
  To the Saṅgha I go for refuge.
\end{english-verses}

\begin{pali-hang-continued}
  Dutiyam'pi buddhaṁ saraṇaṁ gacchāmi\\
  Dutiyam'pi dhammaṁ saraṇaṁ gacchāmi\\
  Dutiyam'pi saṅghaṁ saraṇaṁ gacchāmi
\end{pali-hang-continued}

\begin{english-verses}
  A second time, to the Buddha I go for refuge.\\
  A second time, to the Dhamma I go for refuge.\\
  A second time, to the Saṅgha I go for refuge.
\end{english-verses}

\begin{pali-hang-continued}
  Tatiyam'pi buddhaṁ saraṇaṁ gacchāmi\\
  Tatiyam'pi dhammaṁ saraṇaṁ gacchāmi\\
  Tatiyam'pi saṅghaṁ saraṇaṁ gacchāmi
\end{pali-hang-continued}

\begin{english-verses}
  A third time, to the Buddha I go for refuge.\\
  A third time, to the Dhamma I go for refuge.\\
  A third time, to the Saṅgha I go for refuge.
\end{english-verses}

\suttaRef{[Kp 1]}



\begingroup
\setsechook{%
  % New page for each section.
  \clearpage%
  % Empty the default section number printing, so that we can handle it.
  \setsecnumformat{}%
}

\sectionSubTitle{Nama-kāra-siddhi-gāthā}
\section{Verses on Triumph by Paying Homage}
\label{triumph-by-paying-homage}

\begin{pali-hang-together}
  \anglebracketleft\ \hspace{-0.5mm}Yo cakkhumā \hspace{-0.5mm}\anglebracketright\  moha-malāpakaṭṭho\\
  Sāmaṁ va buddho sugato vimutto\\
  Mārassa pāsā vinimocayanto\\
  Pāpesi khemaṁ janataṁ vineyyaṁ\\
  Buddhaṁ varan'taṁ sirasā namāmi\\
  Lokassa nāthañ'ca vināyakañ'ca\\
  Tan'tejasā te jaya-siddhi hotu\\
  Sabb'antarāyā ca vināsam'entu
\end{pali-hang-together}

\begin{english-verses}
  The One with Vision, with the stain of delusion removed,\\
  Self-awakened, well-gone, \& released,\\
  Freed from the snares of mortal temptation,\\
  He caused tameable people to reach security.\\
  \begin{english-hang-together-verses}
    I pay homage with my head to that excellent Buddha, the protector \& mentor for the world,\\
  \end{english-hang-together-verses}
  By the power of this, may you have triumph \& success,\\
  And may all your dangers be destroyed.
\end{english-verses}

\begin{pali-hang-continued}
  Dhammo dhajo yo viya tassa satthu\\
  Dassesi lokassa visuddhi-maggaṁ\\
  Niyyāniko dhamma-dharassa dhārī\\
  Sāt'āvaho santi-karo suciṇṇo\\
  Dhammaṁ varan'taṁ sirasā namāmi\\
  Moha-ppadālaṁ upasanta-dāhaṁ\\
  Tan'tejasā te jaya-siddhi hotu\\
  Sabb'antarāyā ca vināsam'entu
\end{pali-hang-continued}

\ifafiveversion\clearpage\fi

\begin{english-verses}
  The Teacher's Dhamma, like a banner,\\
  Shows the path of purity to the world.\\
  Leading out, upholding those who uphold it\\
  Rightly accomplished, it brings pleasure, makes peace.\\
  I pay homage with my head to that excellent Dhamma,\\
  Which pierces delusion and makes fever grow calm.\\
  By the power of this, may you have triumph \& success,\\
  And may all your dangers be destroyed.
\end{english-verses}

\begin{pali-hang-continued}
  Saddhamma-senā sugat'ānugo yo\\
  Lokassa pāp'ūpakilesa-jetā\\
  Santo sayaṁ santi-niyojako ca\\
  Svākkhāta-dhammaṁ viditaṁ karoti\\
  Saṅghaṁ varan'taṁ sirasā namāmi\\
  Buddh'ānubuddhaṁ sama-sīla-diṭṭhiṁ\\
  Tan'tejasā te jaya-siddhi hotu\\
  Sabb'antarāyā ca vināsam'entu
\end{pali-hang-continued}

\begin{english-verses}
  The True Dhamma's army, following the One Well-Gone,\\
  Is victor over the evils \& corruptions of the world.\\
  Virtuous, unifying itself in peace,\\
  And makes the well-taught Dhamma be known.\\
  I pay homage with my head to that excellent Saṅgha,\\
  Awakened after the Awakened, harmonious in virtue \& view.\\
  By the power of this, may you have triumph \& success,\\
  And may all your dangers be destroyed.
\end{english-verses}

\suttaRef{[Thai]}



\sectionSubTitle{Namo-kāra-aṭṭhaka}
\section{Paying Homage to the Octad}
\label{paying-homage-to-octad}

\begin{pali-hang-together}
  \anglebracketleft\ \hspace{-0.5mm}Namo arahato \hspace{-0.5mm}\anglebracketright\ Sammā-\\
  Sambuddhassa mah'esino\\
  Namo uttama-dhammassa\\
  Svākkhātass'eva ten'idha
\end{pali-hang-together}

\begin{english-verses}
  Homage to the Great Seer, the Worthy One, Perfectly Self-awakened; Homage to the highest Dhamma, well-taught by him here;
\end{english-verses}

\begin{pali-hang-continued}
  Namo mahā-saṅghassā'pi\\
  Visuddha-sīla-diṭṭhino\\
  Namo omāty'āraddhassa\\
  Ratanattayassa sādhukaṁ
\end{pali-hang-continued}

\begin{english-verses}
  And homage to the Great Saṅgha,\\
  Pure in virtue \& view.\\
  Homage to the Triple Gem\\
  Beginning auspiciously with AUM;
\end{english-verses}

\begin{pali-hang-continued}
  Namo omak'ātītassa\\
  Tassa vatthuttayassa'pi\\
  Namo-kāra-ppabhāvena\\
  Vigacchantu upaddavā\\
\end{pali-hang-continued}

\begin{english-verses}
  And homage to those three objects\\
  That have left base things behind.\\
  By the potency of this homage,\\
  May misfortunes disappear;
\end{english-verses}

\ifbfiveversion\clearpage\fi

\begin{pali-hang-continued}
  Namo-kār'ānubhāvena\\
  Suvatthi hotu sabbadā\\
  Namo-kārassa tejena\\
  Vidhimhi homi tejavā
\end{pali-hang-continued}

\begin{english-verses}
  By the potency of this homage,\\
  May there always be well-being;\\
  By the power of this homage,\\
  May success in this ceremony be mine.
\end{english-verses}

\suttaRef{[Thai]}



\sectionSubTitle{Maṅgala-sutta}
\section{The Discourse on Blessings}
\label{discorse-on-blessings}

\begin{pali-hang-together}
  \anglebracketleft\ \hspace{-0.5mm}Evaṁ me sutaṁ \hspace{-0.5mm}\anglebracketright\ ekaṁ samayaṁ bhagavā sāvatthiyaṁ viharati jetavane anāthapiṇḍikassa ārāme. Atha kho aññatarā devatā abhikkantāya rattiyā abhikkantavaṇṇā kevalakappaṁ jetavanaṁ obhāsetvā yena bhagavā ten'upasaṅkami. Upasaṅkamitvā bhagavantaṁ abhivādetvā ekam'antaṁ aṭṭhāsi. Ekam'antaṁ ṭhitā kho sā devatā bhagavantaṁ gāthāya ajjhabhāsi:
\end{pali-hang-together}

\begin{english-verses}
  Thus have I heard. On one occasion the Blessed One was dwelling at Sāvatthī in Jeta's Grove, Anāthapiṇḍika's Park. Then, when the night had advanced, a certain deity of stunning beauty, having illuminated the entire Jeta's Grove, approached the Blessed One, paid homage to him, stood to one side, and addressed the Blessed One in verse:
\end{english-verses}

\begin{pali-hang-continued}
  Bahū devā manussā ca\\
  Maṅgalāni acintayuṁ\\
  Ākaṅkhamānā sotthānaṁ\\
  Brūhi maṅgalam'uttamaṁ
\end{pali-hang-continued}

\ifninebythirteenversion\clearpage\fi

\begin{english-verses}
  Many devas and human beings\\
  Have reflected on blessings,\\
  Longing for safety,\\
  So declare the highest blessing.
\end{english-verses}

\begin{pali-hang-continued}
  Asevanā ca bālānaṁ\\
  Paṇḍitānañ'ca sevanā\\
  Pūjā ca pūjanīyānaṁ\\
  Etam'maṅgalam'uttamaṁ
\end{pali-hang-continued}

\begin{english-verses}
  Not associating with fools,\\
  Associating with the wise,\\
  And venerating those worthy of veneration:\\
  This is the highest blessing.
\end{english-verses}

\begin{pali-hang-continued}
  Paṭirūpa-desa-vāso ca\\
  Pubbe ca kata-puññatā\\
  Atta-sammā-paṇidhi ca\\
  Etam'maṅgalam'uttamaṁ
\end{pali-hang-continued}

\begin{english-verses}
  Residing in a suitable place\\
  Merit done in the past\\
  And directing oneself rightly\\
  This is the highest blessing
\end{english-verses}

\begin{pali-hang-continued}
  Bāhu-saccañ'ca sippañ'ca\\
  Vinayo ca susikkhito\\
  Subhāsitā ca yā vācā\\
  Etam'maṅgalam'uttamaṁ
\end{pali-hang-continued}

\begin{english-verses}
  Much learning, a craft,\\
  A well-trained discipline,\\
  And well-spoken speech:\\
  This is the highest blessing.
\end{english-verses}

\begin{pali-hang-continued}
  Mātā-pitu upaṭṭhānaṁ\\
  Putta-dārassa saṅgaho\\
  Anākulā ca kammantā\\
  Etam'maṅgalam'uttamaṁ
\end{pali-hang-continued}

\ifbfiveversion\clearpage\fi

\begin{english-verses}
  Serving one's mother and father,\\
  Maintaining a wife and children,\\
  And an honest occupation:\\
  This is the highest blessing.
\end{english-verses}

\begin{pali-hang-continued}
  Dānañ'ca dhamma-cariyā ca\\
  Ñātakānañ'ca saṅgaho\\
  Anavajjāni kammāni\\
  Etam'maṅgalam'uttamaṁ
\end{pali-hang-continued}

\begin{english-verses}
  Giving and righteous conduct,\\
  Assistance to relatives,\\
  Blameless deeds:\\
  This is the highest blessing.
\end{english-verses}

\ifninebythirteenversion\clearpage\fi

\begin{pali-hang-continued}
  Āratī viratī pāpā\\
  Majja-pānā ca saññamo\\
  Appamādo ca dhammesu\\
  Etam'maṅgalam'uttamaṁ
\end{pali-hang-continued}

\begin{english-verses}
  Desisting and abstaining from evil,\\
  Refraining from intoxicating drink,\\
  Heedfulness in good qualities:\\
  This is the highest blessing.
\end{english-verses}

\begin{pali-hang-continued}
  Gāravo ca nivāto ca\\
  Santuṭṭhī ca kataññutā\\
  Kālena dhamma-ssavanaṁ\\
  Etam'maṅgalam'uttamaṁ
\end{pali-hang-continued}

\ifbfiveversion\clearpage\fi

\begin{english-verses}
  Reverence and humility,\\
  Contentment and gratitude,\\
  Timely listening to the Dhamma:\\
  This is the highest blessing.
\end{english-verses}

\begin{pali-hang-continued}
  Khantī ca sovacassatā\\
  Samaṇānañ'ca dassanaṁ\\
  Kālena dhamma-sākacchā\\
  Etam'maṅgalam'uttamaṁ
\end{pali-hang-continued}

\begin{english-verses}
  Patience, being amenable to advice,\\
  The seeing of ascetics,\\
  Timely discussion on the Dhamma:\\
  This is the highest blessing.
\end{english-verses}

\begin{pali-hang-continued}
  Tapo ca brahmacariyañ'ca\\
  Ariya-saccāna-dassanaṁ\\
  Nibbāna-sacchikiriyā ca\\
  Etam'maṅgalam'uttamaṁ
\end{pali-hang-continued}

\begin{english-verses}
  Austerity and the holy life,\\
  Seeing of the noble truths,\\
  And realization of Nibbāna:\\
  This is the highest blessing.
\end{english-verses}

\begin{pali-hang-continued}
  Phuṭṭhassa loka-dhammehi\\
  Cittaṁ yassa na kampati\\
  Asokaṁ virajaṁ khemaṁ\\
  Etam'maṅgalam'uttamaṁ
\end{pali-hang-continued}

\ifbfiveversion\clearpage\fi

\begin{english-verses}
  One whose mind does not shake\\
  When touched by worldly conditions,\\
  Sorrowless, dust-free, secure:\\
  This is the highest blessing.
\end{english-verses}

\begin{pali-hang-continued}
  Etādisāni katvāna\\
  Sabbattham'aparājitā\\
  Sabbattha sotthiṁ gacchanti\\
  Tan'tesaṁ maṅgalam'uttaman'ti
\end{pali-hang-continued}

\begin{english-verses}
  Those who have done these things\\
  Are victorious everywhere;\\
  Everywhere they go safely:\\
  Theirs is that highest blessing.
\end{english-verses}

\suttaRef{[Snp 2.4]}



\sectionSubTitle{Ratana-sutta}
\section{The Discourse on Jewels}
\label{discourse-on-jewels}

\begin{pali-hang-together}
  \anglebracketleft\ \hspace{-0.5mm}Yān'īdha \hspace{-0.5mm}\anglebracketright\ bhūtāni samāgatāni\\
  Bhummāni vā yāni va antalikkhe\\
  Sabb'eva bhūtā sumanā bhavantu\\
  Atho'pi sakkacca suṇantu bhāsitaṁ
\end{pali-hang-together}

\begin{english-verses}
  Whatever beings are gathered here,\\
  Whether of the earth or in the sky,\\
  May all beings indeed be happy\\
  And then listen carefully to what is said.
\end{english-verses}

\begin{pali-hang-continued}
  Tasmā hi bhūtā nisāmetha sabbe\\
  Mettaṁ karotha mānusiyā pajāya\\
  Divā ca ratto ca haranti ye baliṁ\\
  Tasmā hi ne rakkhatha appamattā
\end{pali-hang-continued}

\begin{english-verses}
  Therefore, O beings, all of you listen;\\
  Show loving-kindness to the human population,\\
  Who day and night bring you offerings;\\
  Therefore, being heedful, protect them.
\end{english-verses}

\begin{pali-hang-continued}
  Yaṅ'kiñci vittaṁ idha vā huraṁ vā\\
  Saggesu vā yaṁ ratanaṁ paṇītaṁ\\
  Na no samaṁ atthi tathāgatena\\
  Idam'pi buddhe ratanaṁ paṇītaṁ\\
  Etena saccena suvatthi hotu
\end{pali-hang-continued}

\begin{english-verses}
  Whatever the treasures are here or beyond,\\
  Whatever the precious jewel in the heavens,\\
  There is none equal to the Thus-gone.\\
  In the Buddha is this sublime jewel.\\
  By this truth, may there be well-being.
\end{english-verses}

\begin{pali-hang-continued}
  Khayaṁ virāgaṁ amataṁ paṇītaṁ\\
  Yad'ajjhagā sakya-munī samāhito\\
  Na tena dhammena sam'atthi kiñci\\
  Idam'pi dhamme ratanaṁ paṇītaṁ\\
  Etena saccena suvatthi hotu
\end{pali-hang-continued}

\begin{english-verses}
  Destruction, dispassion, the deathless, the sublime,\\
  Which Sakyamuni, concentrated, attained:\\
  There is nothing equal to that Dhamma.\\
  This too is the sublime gem in the Dhamma:\\
  By this truth, may there be safety!
\end{english-verses}

\begin{pali-hang-continued}
  Yam'buddha-seṭṭho parivaṇṇayī suciṁ\\
  Samādhim'ānantarikañ'ñam'āhu\\
  Samādhinā tena samo na vijjati\\
  Idam'pi dhamme ratanaṁ paṇītaṁ\\
  Etena saccena suvatthi hotu
\end{pali-hang-continued}

\begin{english-verses}
  The purity that the supreme Buddha praised,\\
  Which they call concentration without interval\\
  The equal of that concentration does not exist.\\
  This too is the sublime gem in the Dhamma:\\
  By this truth, may there be safety!
\end{english-verses}

\begin{pali-hang-continued}
  Ye puggalā aṭṭha sataṁ pasatthā\\
  Cattāri etāni yugāni honti\\
  Te dakkhiṇeyyā sugatassa sāvakā\\
  Etesu dinnāni maha-pphalāni\\
  Idam'pi saṅghe ratanaṁ paṇītaṁ\\
  Etena saccena suvatthi hotu
\end{pali-hang-continued}

\begin{english-verses}
  The eight persons praised by the good\\
  Constitute these four pairs.\\
  These, worthy of offerings, are the Fortunate One's disciples;\\
  Gifts given to them yield abundant fruit.\\
  This too is the sublime gem in the Sangha:\\
  By this truth, may there be safety!
\end{english-verses}

\begin{pali-hang-continued}
  Ye suppayuttā manasā daḷhena\\
  Nikkāmino gotama-sāsanamhi\\
  Te patti-pattā amataṁ vigayha\\
  Laddhā mudhā nibbutiṁ bhuñjamānā\\
  Idam'pi saṅghe ratanaṁ paṇītaṁ\\
  Etena saccena suvatthi hotu
\end{pali-hang-continued}

\begin{english-verses}
  Those who strived well with a firm mind,\\
  Who are desireless in Gotama's teaching,\\
  Have reached attainment, having plunged into the deathless,\\
  Enjoying perfect peace obtained free of charge.\\
  This too is the sublime gem in the Sangha:\\
  By this truth, may there be safety!
\end{english-verses}

\begin{pali-hang-continued}
  Yath'inda-khīlo paṭhaviṁ sito siyā\\
  Catubbhi vātebhi asampakampiyo\\
  Tath'ūpamaṁ sappurisaṁ vadāmi\\
  Yo ariya-saccāni avecca passati\\
  \ifbfiveversion\clearpage\fi
  Idam'pi saṅghe ratanaṁ paṇītaṁ\\
  Etena saccena suvatthi hotu
\end{pali-hang-continued}

\begin{english-verses}
  As a gate post, planted in the ground,\\
  Would be unshakable by the four winds,\\
  Similarly I speak of the good person\\
  Who, having experienced them, sees the noble truths.\\
  This too is the sublime gem in the Sangha:\\
  By this truth, may there be safety!
\end{english-verses}

\begin{pali-hang-continued}
  Ye ariya-saccāni vibhāvayanti\\
  Gambhīra-paññena sudesitāni\\
  Kiñ'c'āpi te honti bhusa'ppamattā\\
  Na te bhavaṁ aṭṭhamam'ādiyanti\\
  \ifninebythirteenversion\clearpage\fi
  Idam'pi saṅghe ratanaṁ paṇītaṁ\\
  Etena saccena suvatthi hotu
\end{pali-hang-continued}

\begin{english-verses}
  Those who have cognized the noble truths\\
  Well taught by the one of deep wisdom,\\
  Even if they are extremely heedless,\\
  Do not take an eighth existence.\\
  This too is the sublime gem in the Sangha:\\
  By this truth, may there be safety!
\end{english-verses}

\begin{pali-hang-continued}
  Sahā v'assa dassana-sampadāya\\
  Tay'assu dhammā jahitā bhavanti\\
  Sakkāya-diṭṭhi vicikicchitañ'ca\\
  Sīlabbataṁ vā'pi yad'atthi kiñci\\
  Catūh'apāyehi ca vippamutto\\
  Cha c'ābhiṭhānāni abhabbo kātuṁ\\
\ifbfiveversion\clearpage\fi
  Idam'pi saṅghe ratanaṁ paṇītaṁ\\
  Etena saccena suvatthi hotu
\end{pali-hang-continued}

\begin{english-verses}
  Together with one's achievement of vision\\
  Three things are discarded:\\
  The view of the personal entity and doubt,\\
  And whatever good behavior and observances there are.\\
  One is also freed from the four planes of misery\\
  And is incapable of doing six deeds.\\
  \ifninebythirteenversion\clearpage\fi
  This too is the sublime gem in the Sangha:\\
  By this truth, may there be safety!
\end{english-verses}

\begin{pali-hang-continued}
  Kiñ'c'āpi so kammaṁ karoti pāpakaṁ\\
  Kāyena vācā uda cetasā vā\\
  Abhabbo so tassa paṭicchadāya\\
  Abhabbatā diṭṭha-padassa vuttā\\
  Idam'pi saṅghe ratanaṁ paṇītaṁ\\
  Etena saccena suvatthi hotu
\end{pali-hang-continued}

\begin{english-verses}
  Although one does a bad deed\\
  By body, speech, or mind,\\
  One is incapable of concealing it;\\
  Such inability is stated for one who has seen the state.\\
  This too is the sublime gem in the Sangha:\\
  By this truth, may there be safety!
\end{english-verses}

\begin{pali-hang-continued}
  Vana-ppagumbe yathā phussit'agge\\
  Gimhāna-māse paṭhamasmiṁ gimhe\\
  Tath'ūpamaṁ dhamma-varaṁ adesayi\\
  Nibbāna-gāmiṁ paramaṁ hitāya\\
  Idam'pi buddhe ratanaṁ paṇītaṁ\\
  Etena saccena suvatthi hotu
\end{pali-hang-continued}

\begin{english-verses}
  Like a woodland thicket with flowering crests\\
  In a summer month, in the first of the summer,\\
  Just so he taught the excellent Dhamma,\\
  Leading to Nibbāna, for the supreme welfare.\\
  This too is the sublime gem in the Buddha:\\
  By this truth, may there be safety!
\end{english-verses}

\begin{pali-hang-continued}
  Varo varaññū varado var'āharo\\
  Anuttaro dhamma-varaṁ adesayi\\
  Idam'pi buddhe ratanaṁ paṇītaṁ\\
  Etena saccena suvatthi hotu
\end{pali-hang-continued}

\begin{english-verses}
  The excellent one, knower of the excellent,\\
  Giver of the excellent, bringer of the excellent,\\
  The unsurpassed one taught the excellent Dhamma.\\
  This too is the sublime gem in the Buddha:\\
  By this truth, may there be safety!
\end{english-verses}

\begin{pali-hang-continued}
  Khīṇaṁ purāṇaṁ navaṁ n'atthi sambhavaṁ\\
  Viratta-citt'āyatike bhavasmiṁ\\
  Te khīṇa-bījā aviruḷhi-chandā\\
  Nibbanti dhīrā yathā'yaṁ padīpo\\
  Idam'pi saṅghe ratanaṁ paṇītaṁ\\
  Etena saccena suvatthi hotu
\end{pali-hang-continued}

\begin{english-verses}
  The old is destroyed, there is no new origination,\\
  Their minds are dispassionate toward future existence.\\
  With seeds destroyed, with no desire for growth,\\
  Those wise ones are extinguished like this lamp.\\
  \ifninebythirteenversion\clearpage\fi
  This too is the sublime gem in the Sangha:\\
  By this truth, may there be safety!
\end{english-verses}

\begin{pali-hang-continued}
  Yān'īdha bhūtāni samāgatāni\\
  Bhummāni vā yāni va antalikkhe\\
  Tathāgataṁ deva-manussa-pūjitaṁ\\
  Buddhaṁ namassāma suvatthi hotu
\end{pali-hang-continued}

\begin{english-verses}
  Whatever beings are gathered here,\\
  Whether of the earth or in the sky,\\
  We pay homage to the thus-gone Buddha,\\
  Venerated by devas and humans: may there be safety!
\end{english-verses}

\begin{pali-hang-continued}
  Yān'īdha bhūtāni samāgatāni\\
  Bhummāni vā yāni va antalikkhe\\
  Tathāgataṁ deva-manussa-pūjitaṁ\\
  Dhammaṁ namassāma suvatthi hotu
\end{pali-hang-continued}

\begin{english-verses}
  Whatever beings are gathered here,\\
  Whether of the earth or in the sky,\\
  We pay homage to the thus-gone Dhamma,\\
  Venerated by devas and humans: may there be safety!
\end{english-verses}

\begin{pali-hang-continued}
  Yān'īdha bhūtāni samāgatāni\\
  Bhummāni vā yāni va antalikkhe\\
  Tathāgataṁ deva-manussa-pūjitaṁ\\
  Saṅghaṁ namassāma suvatthi hotu
\end{pali-hang-continued}

\begin{english-verses}
  Whatever beings are gathered here,\\
  Whether of the earth or in the sky,\\
  We pay homage to the thus-gone Sangha,\\
  Venerated by devas and humans: may there be safety!
\end{english-verses}

\suttaRef{[Snp 2.1]}



\sectionSubTitle{Karaṇīya-metta-sutta}
\section{The Discourse on the Loving-Kindness \mbox{to}~be~Done}
\label{discourse-on-loving-kindness}

\begin{pali-hang-together}
  \anglebracketleft\ \hspace{-0.5mm}Karaṇīyam \hspace{-0.5mm}\anglebracketright\ 'attha-kusalena\\
  Yan'taṁ santaṁ padaṁ abhisamecca\\
  Sakko ujū ca suhujū ca\\
  Suvaco c'assa mudu anatimānī
\end{pali-hang-together}

\begin{english-verses}
  This is to be done by one skilled in the beneficial,\\
  Having understood the path of peace.\\
  He would be able, upright, very upright,\\
  Obedient, gentle, and not conceited.
\end{english-verses}

\begin{pali-hang-continued}
  Santussako ca subharo ca\\
  Appakicco ca sallahuka-vutti\\
  Sant'indriyo ca nipako ca\\
  Appagabbho kulesu ananugiddho
\end{pali-hang-continued}

\begin{english-verses}
  Contented and easy to support,\\
  Having little duties and a light livelihood;\\
  Calm in faculties and prudent,\\
  Not impudent \& greedily attached to families.
\end{english-verses}

\begin{pali-hang-continued}
  Na ca khuddaṁ samācare kiñci\\
  Yena viññū pare upavadeyyuṁ\\
  Sukhino vā khemino hontu\\
  Sabbe sattā bhavantu sukhit'attā
\end{pali-hang-continued}

\begin{english-verses}
  He would not do the slightest thing,\\
  For which wise others would reproach.\\
  Well and secure may [all beings] be;\\
  May all beings be happy at heart.
\end{english-verses}

\begin{pali-hang-continued}
  Ye keci pāṇa-bhūt'atthi\\
  Tasā vā thāvarā vā anavasesā\\
  Dīghā vā ye mahantā vā\\
  Majjhimā rassakā aṇuka-thūlā
\end{pali-hang-continued}

\begin{english-verses}
  Whatever breathing beings are born,\\
  Whether timid or firm – without remainder\\
  Whether they are long or they are great,\\
  Whether they are medium, short, minute or fat.
\end{english-verses}

\begin{pali-hang-continued}
  Diṭṭhā vā ye ca adiṭṭhā\\
  Ye ca dūre vasanti avidūre\\
  Bhūtā vā sambhavesī vā\\
  Sabbe sattā bhavantu sukhit'attā
\end{pali-hang-continued}

\begin{english-verses}
  Whether they are seen or unseen,\\
  Whether they live far away or not far away;\\
  Already born or seeking rebirth\\
  May all beings be happy at heart.
\end{english-verses}

\begin{pali-hang-continued}
  Na paro paraṁ nikubbetha\\
  N'ātimaññetha katthaci naṁ kiñci\\
  Byārosanā paṭīgha-saññā\\
  N'āññam'aññassa dukkham'iccheyya
\end{pali-hang-continued}

\begin{english-verses}
  Another (i.e. one) would not deceive another,\\
  Nor look down upon anyone anywhere;\\
  \ifbfiveversion\clearpage\fi
  Nor through anger or hateful perception,\\
  Would wish for one another's suffering.
\end{english-verses}

\begin{pali-hang-continued}
  Mātā yathā niyaṁ puttaṁ\\
  Āyusā eka-puttam'anurakkhe\\
  Evam'pi sabba-bhūtesu\\
  Mānasam'bhāvaye aparimāṇaṁ
\end{pali-hang-continued}

\begin{english-verses}
  Just as a mother [her] own child,\\\relax
  [Her] only child, would protect with [her] life; \\
  Thus also, towards all beings,\\
  He would develop the mind without measure.
\end{english-verses}

\begin{pali-hang-continued}
  Mettañ'ca sabba-lokasmiṁ\\
  Mānasam'bhāvaye aparimāṇaṁ\\
  Uddhaṁ adho ca tiriyañ'ca\\
  Asambādhaṁ averaṁ asapattaṁ
\end{pali-hang-continued}

\begin{english-verses}
  And [with] mettā to the whole world,\\
  He would develop the mind without measure;\\
  Above, and below and across,\\
  Unrestricted, without enmity or foe.
\end{english-verses}

\begin{pali-hang-continued}
  Tiṭṭhañ'caraṁ nisinno vā\\
  Sayāno vā yāvat'assa vigata-middho\\
  Etaṁ satiṁ adhiṭṭheyya\\
  Brahmam'etaṁ vihāraṁ idham'āhu
\end{pali-hang-continued}

\begin{english-verses}
  Whether standing, walking, sitting, or reclining,\\
  Whenever he is free from drowsiness,\\
  \ifbfiveversion\clearpage\fi
  He would resolve on that mindfulness\\
  ``That is a lofty dwelling'', here they say.
\end{english-verses}

\begin{pali-hang-continued}
  Diṭṭhiñ'ca anupagamma\\
  Sīlavā dassanena sampanno\\
  Kāmesu vineyya gedhaṁ\\
  Na hi jātu gabbha-seyyaṁ punar-etī'ti
\end{pali-hang-continued}

\begin{english-verses}
  Not having arrived at a [wrong] view,\\
  Being virtuous and possessed of vision,\\
  Having removed greed for sense pleasures,\\
  He never again returns to lie in a womb.
\end{english-verses}

\suttaRef{[Snp 1.8]}



\sectionSubTitle{Khandha-parittaṁ}
\section{Protection of the Aggregates}
\label{protection-of-aggregates}

\begin{pali-hang-together}
  \anglebracketleft\ \hspace{-0.5mm}Virūpakkhehi me mettaṁ \hspace{-0.5mm}\anglebracketright\ \\
  Mettaṁ erāpathehi me\\
  Chabyāputtehi me mettaṁ\\
  Mettaṁ kaṇhāgotamakehi ca
\end{pali-hang-together}

\begin{english-verses}
  With the Virūpakkhas is my loving-kindness.\\
  My loving-kindness is with the Erāpathas.\\
  With the Chabyāputtas is my loving-kindness.\\
  And loving-kindness is with the Kaṇhāgotamakas.
\end{english-verses}

\begin{pali-hang-continued}
  Apādakehi me mettaṁ\\
  Mettaṁ dipādakehi me\\
  Catuppadehi me mettaṁ\\
  Mettaṁ bahuppadehi me
\end{pali-hang-continued}

\begin{english-verses}
  With the footless is my loving-kindness.\\
  My loving-kindness is with the two-footed.\\
  With the four-footed is my loving-kindness.\\
  My loving-kindness is with the many-footed.
\end{english-verses}

\begin{pali-hang-continued}
  Mā maṁ apādako hiṁsi\\
  Mā maṁ hiṁsi dipādako\\
  \ifninebythirteenversion\clearpage\fi
  Mā maṁ catuppado hiṁsi\\
  Mā maṁ hiṁsi bahuppado
\end{pali-hang-continued}

\begin{english-verses}
  May the footless not harm me.\\
  May the two-footed harm me not.\\
  \ifbfiveversion\clearpage\fi
  May the four-footed not harm me.\\
  May the many-footed harm me not.
\end{english-verses}

\begin{pali-hang-continued}
  Sabbe sattā sabbe pāṇā\\
  Sabbe bhūtā ca kevalā\\
  Sabbe bhadrāni passantu\\
  Mā kiñci pāpam'āgamā.
\end{pali-hang-continued}

\begin{english-verses}
  All sentient beings, all who breathe,\\
  All the born — in totality —\\
  May they all meet with good fortune;\\
  May they not come to any evil.
\end{english-verses}

\begin{pali-hang-continued}
  \anglebracketleft\ \hspace{-0.5mm}Appamāṇo buddho \hspace{-0.5mm}\anglebracketright\ \\
  Appamāṇo dhammo\\
  Appamāṇo saṅgho\\
  Pamāṇavantāni siriṁsapāni\\
  Ahi-vicchikā satapadī\\
  Uṇṇānābhī sarabhū mūsikā
\end{pali-hang-continued}

\ifninebythirteenversion\clearpage\fi

\begin{english-verses}
  Measureless is the Buddha,\\
  Measureless is the Dhamma,\\
  Measureless is the Saṅgha.\\
  Measureable are crawling creatures:\\
  Snakes and scorpions and centipedes,\\
  Spiders and lizards and mice and rats.
\end{english-verses}

\begin{pali-hang-continued}
  Katā me rakkhā katā me parittā\\
  Paṭikkamantu bhūtāni\\
  So'haṁ namo bhagavato\\
  Namo sattannaṁ sammā-sambuddhānaṁ.
\end{pali-hang-continued}

\begin{english-verses}
  A safeguard has been made by me,\\
  A protection has been made by me.\\
  Let the already born retreat.\\
  I pay homage to the Blessed One;\\
  Homage to the seven perfectly self-awakened ones.
\end{english-verses}

\suttaRef{[AN 4.67]}



\sectionSubTitle{Buddha-dhamma-saṅgha-guṇā}
\section{Qualities of the Buddha, \mbox{Dhamma,}~and~Sangha}
\label{qualities-of-buddha-dhamma-sangha}

\begin{pali-hang}
  \anglebracketleft\ \hspace{-0.5mm}Iti'pi so bhagavā \hspace{-0.5mm}\anglebracketright\ \\
  Arahaṁ sammā-sambuddho\\
  Vijjā-caraṇa-sampanno\\
  Sugato loka-vidū\\
  Anuttaro purisa-damma-sārathi\\
  Satthā deva-manussānaṁ\\
  Buddho bhagavā'ti
\end{pali-hang}

\begin{english-verses}
  Thus also is the Blessed One\\
  An arahant, fully self-awakened,\\
  Accomplished in knowledge and conduct,\\
  Fortunate, knower of the world,\\
  Unsurpassed leader of persons to be tamed,\\
  Teacher of deities and humans,\\
  Awakened and blessed.
\end{english-verses}

\begin{pali-hang-continued}
  Svākkhāto bhagavatā dhammo\\
  Sandiṭṭhiko akāliko ehi-passiko\\
  Opanayiko paccattaṁ veditabbo viññūhī'ti
\end{pali-hang-continued}

\begin{english-verses}
  The Dhamma is well-expounded by the Blessed One,\\
  Directly visible, immediate, inviting one to come and see,\\
  Applicable, to be personally experienced by the wise.
\end{english-verses}

\begin{pali-hang-continued}
  Supaṭipanno bhagavato sāvaka-saṅgho\\
  Uju-paṭipanno bhagavato sāvaka-saṅgho\\
  Ñāya-paṭipanno bhagavato sāvaka-saṅgho\\
  Sāmīci-paṭipanno bhagavato sāvaka-saṅgho\\
  Yad'idaṁ cattāri purisa-yugāni aṭṭha purisa-puggalā\\
  Esa bhagavato sāvaka-saṅgho\\
  Āhuneyyo pāhuneyyo dakkhiṇeyyo añjali-karaṇīyo\\
  Anuttaraṁ puñña-kkhettaṁ lokassā'ti
\end{pali-hang-continued}

\begin{english-verses}
  \begin{english-hang-first-line}
    Practicing the good way is the community of the Blessed One's disciples;
  \end{english-hang-first-line}
  \begin{english-hang-together-verses}
    Practicing the straight way is the community of the Blessed One's disciples;
  \end{english-hang-together-verses}
  \begin{english-hang-together-verses}
    Practicing the true way is the community of the Blessed One's disciples;
  \end{english-hang-together-verses}
  \begin{english-hang-together-verses}
    Practicing the proper way is the community of the Blessed One's disciples;
  \end{english-hang-together-verses}
  \begin{english-hang-together-verses}
    That is, the four pairs of persons, the eight kinds of individuals.
  \end{english-hang-together-verses}
  \begin{english-hang-together-verses}
    This is the community of the Blessed One's disciples;
  \end{english-hang-together-verses}
  \begin{english-hang-together-verses}
    Worthy of gifts, worthy of hospitality, worthy of offerings, worthy of reverential salutation,
  \end{english-hang-together-verses}
  \begin{english-hang-together-verses}
    The unsurpassed field of merit for the world.
  \end{english-hang-together-verses}
\end{english-verses}

\suttaRef{[SN 11.3]}



\sectionSubTitle{Yaṅ'kiñci ratanaṁ loke}
\section{Whatever Kind of Jewel \mbox{in}~the~World}
\label{whatever-kind-of-jewel}

\begin{pali-hang-together}
  \anglebracketleft\ \hspace{-0.5mm}Yaṅ'kiñci ratanaṁ loke \hspace{-0.5mm}\anglebracketright\ \\
  Vijjati vividhaṁ puthu\\
  Ratanaṁ buddha-samaṁ n'atthi\\
  Tasmā sotthī bhavantu te
\end{pali-hang-together}

\begin{english-verses}
  Whatever kind of jewel in the world\\
  There is found by a human being,\\
  A jewel comparable to the Buddha does not exist;\\
  Therefore may you be blessed.
\end{english-verses}

\begin{pali-hang-continued}
  Yaṅ'kiñci ratanaṁ loke\\
  Vijjati vividhaṁ puthu\\
  Ratanaṁ dhamma-samaṁ n'atthi\\
  Tasmā sotthī bhavantu te
\end{pali-hang-continued}

\begin{english-verses}
  Whatever kind of jewel in the world\\
  There is found by a human being,\\
  A jewel comparable to the Dhamma does not exist;\\
  Therefore may you be blessed.
\end{english-verses}

\begin{pali-hang-continued}
  Yaṅ'kiñci ratanaṁ loke\\
  Vijjati vividhaṁ puthu\\
  Ratanaṁ saṅgha-samaṁ n'atthi\\
  Tasmā sotthī bhavantu te
\end{pali-hang-continued}

\begin{english-verses}
  Whatever kind of jewel in the world\\
  There is found by a human being,\\
  \ifbfiveversion\clearpage\fi
  A jewel comparable to the Saṅgha does not exist;\\
  Therefore may you be blessed.
\end{english-verses}

\begin{pali-hang-continued}
  Sakkatvā buddha-ratanaṁ\\
  \linkdest{endnote127-body}
  Osadhaṁ\ifdigitalversion\makeatletter\hyperlink{endnote127-appendix}\Hy@raisedlink{{\pagenote{%
        \hypertarget{endnote127-appendix}{\hyperlink{endnote127-body}{WPN: ``\textit{Osathaṁ}''. \textit{Osatha} is not a word found in Pāli dictionaries, but ``\textit{osadha}'' (medicine) is. The spelling ``\textit{osatha}'' is due to a faulty transliteration, influenced by Thai style of Pāli pronunciation, where ``d'' often becomes ``t''. The \textit{Mahā-jaya-maṅgala-gāthā} was originally composed in Sri Lanka. The Sri Lankan version indeed speaks of ``\textit{osadha}'', thus confirming above explanation for the spelling error.}}}}}\makeatother\fi
  uttamaṁ varaṁ\\
  Hitaṁ deva-manussānaṁ\\
  Buddha-tejena sotthinā\\
  Nassant'upaddavā sabbe\\
  Dukkhā vūpasamentu te
\end{pali-hang-continued}

\begin{english-verses}
  Having revered the jewel of the Buddha,\\
  The highest, most excellent medicine,\\
  The welfare of human \& heavenly beings:\\
  Through the Buddha's might \& safety\\
  May all obstacles vanish,\\
  May your sufferings grow totally calm.
\end{english-verses}

\begin{pali-hang-continued}
  Sakkatvā dhamma-ratanaṁ\\
  Osadhaṁ uttamaṁ varaṁ\\
  Pariḷāh'ūpasamanaṁ\\
  Dhamma-tejena sotthinā\\
  \ifninebythirteenversion\clearpage\fi
  Nassant'upaddavā sabbe\\
  Bhayā vūpasamentu te
\end{pali-hang-continued}

\begin{english-verses}
  Having revered the jewel of the Dhamma,\\
  The highest, most excellent medicine,\\
  The stiller of feverish passion:\\
  Through the Dhamma's might \& safety\\
  May all obstacles vanish,\\
  May your fears grow totally calm.
\end{english-verses}

\begin{pali-hang-continued}
  Sakkatvā saṅgha-ratanaṁ\\
  Osadhaṁ uttamaṁ varaṁ\\
  Āhuneyyaṁ pāhuneyyaṁ\\
  Saṅgha-tejena sotthinā\\
  Nassant'upaddavā sabbe\\
  Rogā vūpasamentu te
\end{pali-hang-continued}

\begin{english-verses}
  Having revered the jewel of the Saṅgha,\\
  The highest, most excellent medicine,\\
  Worthy of gifts, worthy of hospitality:\\
  Through the Saṅgha's might \& safety\\
  May all obstacles vanish,\\
  May your diseases grow totally calm.
\end{english-verses}

\suttaRef{[MJG]}



\sectionSubTitle{Bojjh'aṅga-parittaṁ}
\section{Protection of the Factors of Awakening}
\label{protection-of-factors-of-awakening}

\begin{pali-hang-together}
  \anglebracketleft\ \hspace{-0.5mm}Bojjh'aṅgo sati-saṅkhāto \hspace{-0.5mm}\anglebracketright\ \\
  Dhammānaṁ vicayo tathā\\
  Viriyam'pīti-passaddhi\\
  Bojjh'aṅgā ca tathā'pare\\
  Samādh'upekkha-bojjh'aṅgā\\
  Satt'ete sabba-dassinā\\
  Muninā samma'd'akkhātā\\
  Bhāvitā bahulī-katā\\
  Saṁvattanti abhiññāya\\
  Nibbānāya ca bodhiyā\\
  Etena sacca-vajjena\\
  Sotthi te hotu sabbadā
\end{pali-hang-together}

\begin{english-verses}
  The Factors for Awakening include mindfulness,\\
  Investigation of qualities,\\
  Persistence, rapture, \& serenity,\\
  Plus concentration \& equanimity factors for awakening.\\
  These seven, which the All-seeing Sage has perfectly taught,\\
\begin{english-hang-together-verses}
  When developed \& matured bring about heightened knowledge, liberation, \& awakening.\\
\end{english-hang-together-verses}
 \ifninebythirteenversion\clearpage\fi
\begin{english-hang-together-verses}
  By the saying of this truth,\\
\end{english-hang-together-verses}
\begin{english-hang-together-verses}
  May you always be well.
\end{english-hang-together-verses}
\end{english-verses}

\begin{pali-hang-continued}
  Ekasmiṁ samaye nātho\\
  Moggallānañ'ca kassapaṁ\\
  Gilāne dukkhite disvā\\
  Bojjh'aṅge satta desayi\\
  Te ca taṁ abhinanditvā\\
  Rogā mucciṁsu taṅ'khaṇe\\
  Etena sacca-vajjena\\
  Sotthi te hotu sabbadā
\end{pali-hang-continued}

\begin{english-verses}
\begin{english-hang-first-line}
  At one time, our Protector seeing that Moggallana \& Kassapa\\
\end{english-hang-first-line}
\begin{english-hang-together-verses}
  Were sick \& in pain, taught them the Seven Factors for Awakening.\\
\end{english-hang-together-verses}
\begin{english-hang-together-verses}
  They, delighting in that, were instantly freed from their illness.\\
\end{english-hang-together-verses}
\begin{english-hang-together-verses}
  By the saying of this truth, may you always be well.
\end{english-hang-together-verses}
\end{english-verses}

\begin{pali-hang-continued}
  Ekadā dhamma-rājā'pi\\
  Gelaññen'ābhipīḷito\\
  Cundattherena taññ'eva\\
  Bhaṇāpetvāna sādaraṁ\\
  Sammoditvā ca ābādhā\\
  Tamhā vuṭṭhāsi ṭhānaso\\
  Etena sacca-vajjena\\
  Sotthi te hotu sabbadā
\end{pali-hang-continued}

\begin{english-verses}
\begin{english-hang-first-line}
  Once, when the Dhamma King was afflicted with fever,\\
\end{english-hang-first-line}
\begin{english-hang-together-verses}
  He had the elder Cunda recite that very teaching with devotion.\\
\end{english-hang-together-verses}
\begin{english-hang-together-verses}
  And as he approved, he rose up from that disease.\\
\end{english-hang-together-verses}
\begin{english-hang-together-verses}
  By the saying of this truth, may you always be well.
\end{english-hang-together-verses}
\end{english-verses}

\begin{pali-hang-continued}
  Pahīnā te ca ābādhā\\
  Tiṇṇannam'pi mah'esinaṁ\\
  Magg'āhata-kilesā va\\
  Patt'ānuppatti-dhammataṁ\\
  \ifbfiveversion\clearpage\fi
  Etena sacca-vajjena\\
  Sotthi te hotu sabbadā
\end{pali-hang-continued}

\begin{english-verses}
  Those diseases were abandoned by the three great seers,\\
  Just as defilements are demolished by the Path\\
  In accordance with step-by-step attainment.\\
  By the saying of this truth, may you always be well.
\end{english-verses}

\suttaRef{[Thai]}



\sectionSubTitle{Abhaya-parittaṁ}
\section{Protection for Safety}
\label{protection-for-safety}

\begin{pali-hang-together}
  \anglebracketleft\ \hspace{-0.5mm}Yan'dunnimittaṁ \hspace{-0.5mm}\anglebracketright\ avamaṅgalañ'ca\\
  Yo c'āmanāpo sakuṇassa saddo\\
  Pāpaggaho dussupinaṁ akantaṁ\\
  Buddh'ānubhāvena vināsam'entu
\end{pali-hang-together}

\begin{english-verses}
  May bad omens, inauspiciousness,\\
  Undesirable sounds of birds,\\
  Unlucky planets and unpleasant bad dreams\\
  Go to ruin by the power of the Buddha.
\end{english-verses}

\begin{pali-hang-continued}
  Yan'dunnimittaṁ avamaṅgalañ'ca\\
  Yo c'āmanāpo sakuṇassa saddo\\
  Pāpaggaho dussupinaṁ akantaṁ\\
  Dhamm'ānubhāvena vināsam'entu
\end{pali-hang-continued}

\begin{english-verses}
  May bad omens, inauspiciousness,\\
  Undesirable sounds of birds,\\
  Unlucky planets and unpleasant bad dreams\\
  Go to ruin by the power of the Dhamma.
\end{english-verses}

\begin{pali-hang-continued}
  Yan'dunnimittaṁ avamaṅgalañ'ca\\
  Yo c'āmanāpo sakuṇassa saddo\\
  \ifninebythirteenversion\clearpage\fi
  Pāpaggaho dussupinaṁ akantaṁ\\
  Saṅgh'ānubhāvena vināsam'entu
\end{pali-hang-continued}

\begin{english-verses}
  May bad omens, inauspiciousness,\\
  Undesirable sounds of birds,\\
  \ifbfiveversion\clearpage\fi
  Unlucky planets and unpleasant bad dreams\\
  Go to ruin by the power of the Saṅgha.
\end{english-verses}

\suttaRef{[Trad]}



\sectionSubTitle{Devatā-uyyojana-gāthā}
\section{Verses on Dismissing Evil Spirits}
\label{dismissing-the-devatas}

\begin{pali-hang-together}
  \anglebracketleft\ \hspace{-0.5mm}Dukkhappattā \hspace{-0.5mm}\anglebracketright\ ca niddukkhā\\
  Bhayappattā ca nibbhayā\\
  Sokappattā ca nissokā\\
  Hontu sabbe'pi pāṇino
\end{pali-hang-together}

\begin{english-verses}
  [May] sufferers be without suffering,\\
  The fear-struck be without fear,\\
  The grief-stricken be without grief\\\relax
  [Thus] may all beings be.
\end{english-verses}

\begin{pali-hang-continued}
  Ettāvatā ca amhehi\\
  Sambhataṁ puñña-sampadaṁ\\
  \linkdest{endnote128-body}
  Sabbe devā anumodantu\ifdigitalversion\makeatletter\hyperlink{endnote128-appendix}\Hy@raisedlink{{\pagenote{%
        \hypertarget{endnote128-appendix}{\hyperlink{endnote128-body}{WPN: ``\textit{devānumodantu}''}}}}}\makeatother\fi\\
  Sabba-sampatti-siddhiyā
\end{pali-hang-continued}

\begin{english-verses}
  To the extent that all of us\\
  Have accumulated a wealth of merits;\\
  In this may all devas rejoice,\\
  For the attainment of all fortunes.
\end{english-verses}

\begin{pali-hang-continued}
  Dānaṁ dadantu saddhāya\\
  Sīlaṁ rakkhantu sabbadā\\
  \ifninebythirteenversion\clearpage\fi
  Bhāvan'ābhiratā hontu\\
  Gacchantu devatā-gatā
\end{pali-hang-continued}

\begin{english-verses}
  May they give gifts with faith.\\
  May they guard moral precepts always.\\
  \ifbfiveversion\clearpage\fi
  May they delight in mind-development.\\
  May the deities who have come go [back].
\end{english-verses}

%% \vspace{-0.99em}

\begin{pali-hang-continued}
  \anglebracketleft\ \hspace{-0.5mm}Sabbe buddhā \hspace{-0.5mm}\anglebracketright\ balappattā\\
  Paccekānañ'ca yaṁ balaṁ\\
  Arahantānañ'ca tejena\\
  Rakkhaṁ bandhāmi sabbaso
\end{pali-hang-continued}

\begin{english-verses}
  All Buddhas possess [supernormal] strength.\\
  And [there is] the power of Paccekabuddhas.\\
  By the power of the arahants too,\\
  I create protection for all times.
\end{english-verses}

\suttaRef{[MJG]}



\sectionSubTitle{Jaya-maṅgala-aṭṭha-gāthā}
\section{Verses on the Eight Victory Blessings}
\label{eight-victory-blessings}

\begin{pali-hang-together}
  \anglebracketleft\ \hspace{-0.5mm}Bāhuṁ \hspace{-0.5mm}\anglebracketright\ sahassam'abhinimmita sāvudhan'taṁ\\
  Grīmekhalaṁ udita-ghora-sasena-māraṁ\\
  \linkdest{endnote129-body}
  Dān'ādi-dhamma-vidhinā jitavā\makeatletter\hyperlink{endnote129-appendix}\Hy@raisedlink{{\pagenote{%
        \hypertarget{endnote129-appendix}{\hyperlink{endnote129-body}{Here and in all subsequent verses we find the word ``\textit{jitavā}''. The standard spelling would be ``\textit{jitvā}'' (abs. of \textit{jināti}; having conquered). In contrast ``\textit{jitavā}'' is a transliteration based on Thai spelling and pronunciation of Pāli, which has a tendency to insert the letter ``a'' between two consonants such as -\textit{tvā}, thus making it into -\textit{tavā}. It is by spelling anomalies like this, that Pāli scholars can determine the place of origin and age of certain Pāli texts.}}}}}\makeatother
  mun'indo\\
  Tan'tejasā bhavatu te jaya-maṅgalāni
\end{pali-hang-together}

\begin{english-verses}
  \begin{english-hang-first-line}
  Creating a form with 1,000 arms, each equipped with a weapon,
  \end{english-hang-first-line}
  \begin{english-hang-together-verses}
    Māra, on the elephant Girimekhala, uttered a frightening roar together with his troops.
  \end{english-hang-together-verses}
  \begin{english-hang-together-verses}
    The Lord of Sages defeated him by means of such qualities as generosity:
  \end{english-hang-together-verses}
  By the power of this, may you have victory blessings.
\end{english-verses}

\begin{pali-hang-continued}
  Mār'ātirekam'abhiyujjhita-sabba-rattiṁ\\
  Ghoram'pan'āḷavakam'akkhama-thaddha-yakkhaṁ\\
  Khantī-sudanta-vidhinā jitavā mun'indo\\
  Tan'tejasā bhavatu te jaya-maṅgalāni
\end{pali-hang-continued}

\begin{english-verses}
  Even more frightful than Māra making war all night\\
  Was Āḷavaka, the arrogant unstable ogre.\\
  \ifninebythirteenversion\clearpage\fi
  \begin{english-hang-together-verses}
  The Lord of Sages defeated him by means of well-trained endurance:\\
  \end{english-hang-together-verses}
  \begin{english-hang-together-verses}
  By the power of this, may you have victory blessings.
  \end{english-hang-together-verses}
\end{english-verses}

\begin{pali-hang-continued}
  Nāḷāgiriṁ gaja-varaṁ atimatta-bhūtaṁ\\
  Dāv'aggi-cakkam'asanīva sudāruṇan'taṁ\\
  Mett'ambu-seka-vidhinā jitavā mun'indo\\
  Tan'tejasā bhavatu te jaya-maṅgalāni\\
\end{pali-hang-continued}

\ifbfiveversion\clearpage\fi
\begin{english-verses}
  Nāḷāgiri, the excellent elephant, when maddened,\\
  \begin{english-hang-together-verses}
  Was very horrific, like a forest fire, a flaming discus, a lightning bolt.\\
  \end{english-hang-together-verses}
  \begin{english-hang-together-verses}
  The Lord of Sages defeated him by sprinkling the water of good will:\\
  \end{english-hang-together-verses}
  \begin{english-hang-together-verses}
  By the power of this, may you have victory blessings.
  \end{english-hang-together-verses}
\end{english-verses}

\begin{pali-hang-continued}
  Ukkhitta-khaggam'atihattha-sudāruṇan'taṁ\\
  Dhāvan'ti-yojana-path'aṅguli-mālavantaṁ\\
  Iddhī'bhisaṅkhata-mano jitavā mun'indo\\
  Tan'tejasā bhavatu te jaya-maṅgalāni
\end{pali-hang-continued}

\begin{english-verses}
  Very horrific, with a sword upraised in his expert hand,\\
  Garlanded-with-Fingers ran three leagues along the path.\\
  \begin{english-hang-together-verses}
  The Lord of Sages defeated him with mind-fashioned marvels:\\
  \end{english-hang-together-verses}
  \begin{english-hang-together-verses}
  By the power of this, may you have victory blessings.
  \end{english-hang-together-verses}
\end{english-verses}

\begin{pali-hang-continued}
  Katvāna kaṭṭham'udaraṁ iva gabbhinīyā\\
  Ciñcāya duṭṭha-vacanaṁ jana-kāya majjhe\\
  Santena soma-vidhinā jitavā mun'indo\\
  Tan'tejasā bhavatu te jaya-maṅgalāni
\end{pali-hang-continued}

\begin{english-verses}
  Having made a wooden belly to appear pregnant,\\
  Ciñca made a lewd accusation in the midst of the gathering.\\
  \begin{english-hang-together-verses}
  The Lord of Sages defeated her with peaceful, gracious means:\\
  \end{english-hang-together-verses}
  \begin{english-hang-together-verses}
  By the power of this, may you have victory blessings.
  \end{english-hang-together-verses}
\end{english-verses}

\begin{pali-hang-continued}
  Saccaṁ vihāya mati-saccaka-vāda-ketuṁ\\
  Vād'ābhiropita-manaṁ ati-andha-bhūtaṁ\\
  Paññā-padīpa-jalito jitavā mun'indo\\
  Tan'tejasā bhavatu te jaya-maṅgalāni
\end{pali-hang-continued}

\begin{english-verses}
  Saccaka, whose provocative views had abandoned the truth,\\
  Delighting in argument, had become thoroughly blind.\\
  \begin{english-hang-together-verses}
  The Lord of Sages defeated him with the light of discernment:\\
  \end{english-hang-together-verses}
  \begin{english-hang-together-verses}
  By the power of this, may you have victory blessings.
  \end{english-hang-together-verses}
\end{english-verses}

\begin{pali-hang-continued}
  Nandopananda-bhujagaṁ vibudhaṁ mah'iddhiṁ\\
  Puttena thera-bhujagena dam'āpayanto\\
  Iddh'ūpadesa-vidhinā jitavā mun'indo\\
  Tan'tejasā bhavatu te jaya-maṅgalāni
\end{pali-hang-continued}

\begin{english-verses}
  \begin{english-hang-first-line}
  Nandopananda was a serpent with great power but wrong views.\\
  \end{english-hang-first-line}
  \begin{english-hang-together-verses}
  The Lord of Sages defeated him by means of a display of marvels,\\
  \end{english-hang-together-verses}
  \begin{english-hang-together-verses}
  Sending his son, the serpent-elder, to tame him:\\
  \end{english-hang-together-verses}
  \begin{english-hang-together-verses}
  By the power of this, may you have victory blessings.
  \end{english-hang-together-verses}
\end{english-verses}

\begin{pali-hang-continued}
  Duggāha-diṭṭhi-bhujagena sudaṭṭha-hatthaṁ\\
  Brahmaṁ visuddhi-jutim'iddhi-bak'ābhidhānaṁ\\
  Ñāṇāgadena vidhinā jitavā mun'indo\\
  Tan'tejasā bhavatu te jaya-maṅgalāni
\end{pali-hang-continued}

\begin{english-verses}
  His hands bound tight by the serpent of wrongly held views,\\
  \begin{english-hang-together-verses}
  Baka, the Brahma, thought himself pure in his radiance \& power.\\
  \end{english-hang-together-verses}
  \begin{english-hang-together-verses}
  The Lord of Sages defeated him by means of his words of knowledge:\\
  \end{english-hang-together-verses}
  \begin{english-hang-together-verses}
  By the power of this, may you have victory blessings.
  \end{english-hang-together-verses}
\end{english-verses}

\begin{pali-hang-continued}
  Etā'pi buddha-jaya-maṅgala-aṭṭha-gāthā\\
  Yo vācano dina-dine sarate'm'atandī\\
  Hitvān'aneka-vividhāni c'upaddavāni\\
  Mokkhaṁ sukhaṁ adhigameyya naro sapañño
\end{pali-hang-continued}

\begin{english-verses}
  These eight verses of the Buddha's victory blessings:
  \begin{english-hang-together-verses}
    Whatever person of discernment recites or recalls them day after day without lapsing,
  \end{english-hang-together-verses}
  \ifbfiveversion\clearpage\fi
  Destroying all kinds of obstacles,\\
  Will attain emancipation \& happiness.
\end{english-verses}
%TODO 1 single orphan

\suttaRef{[Trad]}



\sectionSubTitle{Jaya-parittaṁ}
\section{Protection for Victory}
\label{protection-for-victory}

\begin{pali-hang-together}
  \anglebracketleft\ \hspace{-0.5mm}Mahā-kāruṇiko \hspace{-0.5mm}\anglebracketright\ nātho\\
  Hitāya sabba-pāṇinaṁ\\
  Pūretvā pāramī sabbā\\
  Patto sambodhim'uttamaṁ\\
  Etena sacca-vajjena\\
  Hotu te jaya-maṅgalaṁ
\end{pali-hang-together}

\begin{english-verses}
  The Great Compassionate Protector,\\
  For the benefit of all beings,\\
  Fulfilled all the perfections\\
  And reached the highest awakening.\\
  By this utterance of truth,\\
  May there be auspicious victory for you.
\end{english-verses}

\begin{pali-hang-continued}
  Jayanto bodhiyā mūle\\
  Sakyānaṁ nandi-vaḍḍhano\\
  Evaṁ tvaṁ vijayo hohi\\
  Jayassu jaya-maṅgale
\end{pali-hang-continued}

\begin{english-verses}
  Victorious at the foot of the Bodhi tree\\
  Was the joy-enhancer of the Sakyas.\\
  Even so, may there be victory.\\
  May you attain auspicious victory.
\end{english-verses}

\begin{pali-hang-continued}
  Aparājita-pallaṅke\\
  Sīse paṭhavi-pokkhare\\
  \ifbfiveversion\clearpage\fi
  Abhiseke sabba-buddhānaṁ\\
  Aggappatto pamodati
\end{pali-hang-continued}

\begin{english-verses}
  At the invincible seat,\\
  The best on Earth,\\
  The consecration place of all Buddhas,\\
  Having reached the highest, he rejoices.
\end{english-verses}

\ifbfiveversion\vspace{-0.5em}\fi
\suttaRef{[MJG]}

\begin{pali-hang-continued}
  Sunakkhattaṁ sumaṅgalaṁ\\
  Supabhātaṁ suhuṭṭhitaṁ\\
  Sukhaṇo sumuhutto ca\\
  Suyiṭṭhaṁ brahmacārisu
\end{pali-hang-continued}

\begin{english-verses}
  Good constellations, good blessings,\\
  Good daybreak, good waking,\\
  Good moment and good time\\
  When offerings are made to holy practitioners.
\end{english-verses}

\begin{pali-hang-continued}
  Padakkhiṇaṁ kāya-kammaṁ\\
  Vācā-kammaṁ padakkhiṇaṁ\\
  Padakkhiṇaṁ mano-kammaṁ\\
  \linkdest{endnote130-body}
  Paṇīdhi te padakkhiṇe\ifdigitalversion\makeatletter\hyperlink{endnote130-appendix}\Hy@raisedlink{{\pagenote{%
        \hypertarget{endnote130-appendix}{\hyperlink{endnote130-body}{WPN: ``\textit{paṇidhi}'' and ``\textit{padakkhiṇā}''}}}}}\makeatother\fi\\
  Padakkhiṇāni katvāna\\
  Labhant'atthe padakkhiṇe
\end{pali-hang-continued}

\begin{english-verses}
  Felicitous is bodily kamma\\
  Verbal kamma is felicitous\\
  Felicitous is mental kamma\\
  When aspiring for felicity.\\
  Having done the felicitous\\
  They get felicitous rewards
\end{english-verses}

\ifbfiveversion\vspace{-1.1em}\fi
\suttaRef{[AN 3.155]}



\sectionSubTitle{Bhavatu-sabba-maṅgalaṁ}
\section{May Every Blessing Come to Be}
\label{may-every-blessing-come-to-be-protective}

\begin{pali-hang-together}
  \anglebracketleft\ \hspace{-0.5mm}Bhavatu sabba-maṅgalaṁ \hspace{-0.5mm}\anglebracketright\ \\
  Rakkhantu sabba-devatā\\
  Sabba-buddh'ānubhāvena\\
  Sadā sotthī bhavantu te
\end{pali-hang-together}

\begin{english-verses}
  May every blessing come to be.\\
  And all good spirits guard you well.\\
  With all the power of the Buddha,\\
  May you always be at ease.
\end{english-verses}

\begin{pali-hang-continued}
  Bhavatu sabba-maṅgalaṁ\\
  Rakkhantu sabba-devatā\\
  Sabba-dhamm'ānubhāvena\\
  Sadā sotthī bhavantu te
\end{pali-hang-continued}

\begin{english-verses}
  May every blessing come to be.\\
  And all good spirits guard you well.\\
  With all the power of the Dhamma,\\
  May you always be at ease.
\end{english-verses}

\begin{pali-hang-continued}
  Bhavatu sabba-maṅgalaṁ\\
  Rakkhantu sabba-devatā\\
  \ifninebythirteenversion\clearpage\fi
  Sabba-saṅgh'ānubhāvena\\
  Sadā sotthī \breathmark\ bhavantu te
\end{pali-hang-continued}

\begin{english-verses}
  May every blessing come to be.\\
  And all good spirits guard you well.\\
  \ifbfiveversion\clearpage\fi
  With all the power of the Sangha,\\
  May you always be at ease.
\end{english-verses}

\suttaRef{[Trad]}

\ifdigitalversion\bottomNav{sharing-merits-departed}\fi

