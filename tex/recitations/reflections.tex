
\ifdigitalversion
  \chapterOpeningPage{reflections-compressed.jpg}
\else
  \chapterOpeningPage{reflections.jpg}
\fi

\chapter{Reflections}



\sectionSubTitle{Cattāro parikkhārā}
\section{The Four Requisites}
\label{four-requisites}

\begin{leader}
  \anglebracketleft\ \hspace{-0.5mm}Handa mayaṁ taṅ'khaṇika-paccavekkhaṇa-pāṭhaṁ bhaṇāmase \hspace{-0.5mm}\anglebracketright\
\end{leader}

Paṭisaṅkhā yoniso cīvaraṁ paṭisevāmi\\
Yāva'd'eva sītassa paṭighātāya\\
Uṇhassa paṭighātāya\\
Ḍaṁsa-makasa-vāt'ātapa-siriṁsapa-samphassānaṁ paṭighātāya\\
Yāva'd'eva hirikopina-paṭicchādan'atthaṁ

\begin{english-keepwith}
  Wisely reflecting \breathmark\ I use the robe
  \begin{english-hangtogether}
    Only to ward off cold \breathmark\ to ward off heat \breathmark\ to ward off the touch of flies \breathmark\ mosquitoes wind burning and creeping things
  \end{english-hangtogether}
  Only for the sake of modesty
\end{english-keepwith}

Paṭisaṅkhā yoniso piṇḍapātaṁ paṭisevāmi\\
N'eva davāya na madāya na maṇḍanāya na vibhūsanāya
\begin{pali-hangtogether}
  Yāva'd'eva imassa kāyassa ṭhitiyā yāpanāya vihiṁs'ūparatiyā brahmacariy'ānuggahāya
\end{pali-hangtogether}
\begin{pali-hangtogether}
  Iti purāṇañ'ca vedanaṁ paṭihaṅkhāmi navañ'ca vedanaṁ na uppādessāmi
\end{pali-hangtogether}
Yātrā ca me bhavissati anavajjatā ca phāsuvihāro cā'ti

\begin{english-keepwith}
  Wisely reflecting \breathmark\ I use almsfood
  \begin{english-hangtogether}
    Not for fun \breathmark\ not for pleasure \breathmark\ not for fattening \breathmark\ not for beautification
  \end{english-hangtogether}
  Only for the maintenance and nourishment of this body\\
  For keeping it healthy \breathmark\ for helping with the holy life\\
  \ifasixversion\clearpage\fi
  Thinking thus: ``I will allay hunger without overeating\\
  So that I may continue to live blamelessly and at ease''
\end{english-keepwith}

\ifafiveversion\clearpage\fi

Paṭisaṅkhā yoniso sen'āsanaṁ paṭisevāmi\\
Yāva'd'eva sītassa paṭighātāya\\
Uṇhassa paṭighātāya\\
Ḍaṁsa-makasa-vāt'ātapa-siriṁsapa-samphassānaṁ paṭighātāya\\
Yāva'd'eva utuparissaya-vinodanaṁ paṭisallān'ārām'atthaṁ

\begin{english-keepwith}
  Wisely reflecting \breathmark\ I use the lodging
  \begin{english-hangtogether}
    Only to ward off cold \breathmark\ to ward off heat \breathmark\ to ward off the touch of flies \breathmark\ mosquitoes wind burning and creeping things
  \end{english-hangtogether}
  \begin{english-hangtogether}
    Only to remove the danger from weather \breathmark\ and for living in seclusion
  \end{english-hangtogether}
\end{english-keepwith}

Paṭisaṅkhā yoniso gilāna-paccaya-bhesajja-parikkhāraṁ paṭisevāmi\\
Yāva'd'eva uppannānaṁ veyyābādhikānaṁ vedanānaṁ paṭighātāya\\
Abyāpajjha-paramatāyā'ti

\begin{english-verses}
  Wisely reflecting \breathmark\ I use supports for the sick and medicinal requisites\\
  Only to ward off painful feelings that have arisen\\
  For the maximum freedom from disease
\end{english-verses}

\suttaRef{[MN 2]}

\bottomNav{five-reflections}



\sectionSubTitle{Āhāra-paṭikūla-paccavekkhaṇa-pāṭho}
\section{The Repulsiveness of Food}
\label{repulsiveness-of-food}

\begin{leader}
  \anglebracketleft\ \hspace{-0.5mm}Handa mayaṁ āhāra-paṭikūla-paccavekkhaṇa-pāṭhaṁ\\ bhaṇāmase \hspace{-0.5mm}\anglebracketright\
\end{leader}

\begin{pali-hang}
  Āhāre paṭikūla-saññā-paricitena bhikkhave \breathmark\ bhikkhuno cetasā bahulaṁ viharato
\end{pali-hang}

\begin{english}
  When a bhikkhu often dwells with a mind\\
  Accustomed to the perception of the repulsiveness of food
\end{english}

Rasa-taṇhāya cittaṁ patilīyati

\begin{english}
  His mind shrinks away from craving for tastes
\end{english}

Patikuṭati pativattati na sampasāriyati

\begin{english-verses}
  Turns back from it\\
  Rolls away from it\\
  And is not drawn towards it
\end{english-verses}

Upekkhā vā pāṭikulyatā vā saṇṭhāti

\begin{english}
  Either equanimity or disgust become settled in him
\end{english}

\suttaRef{[AN 7.49]}

Sabbo pan'āyaṁ piṇḍapāto ajigucchanīyo

\begin{english}
  None of this almsfood is innately repulsive
\end{english}

Imaṁ pūti-kāyaṁ patvā

\begin{english}
  But touching this unclean body
\end{english}

\ifasixversion\clearpage\fi

Ativiya jigucchanīyo jāyati

\begin{english}
  It becomes disgusting indeed
\end{english}

\suttaRef{[Trad]}

\bottomNav{requisites-for-awakening}



\sectionSubTitle{Mettā-pharaṇa}
\section{Universal Well-Being}
\label{universal-well-being}

\begin{leader}
  \anglebracketleft\ \hspace{-0.5mm}Handa mayaṁ mettāpharaṇaṁ karomase \hspace{-0.5mm}\anglebracketright\
\end{leader}

Ahaṁ sukhito homi\\
Niddukkho homi\\
Avero homi\\
Abyāpajjho homi\\
Anīgho homi\\
Sukhī attānaṁ pariharāmi\\
Sabbe sattā sukhitā hontu\\
Sabbe sattā averā hontu\\
Sabbe sattā abyāpajjhā hontu\\
Sabbe sattā anīghā hontu\\
Sabbe sattā sukhī attānaṁ pariharantu\\
Sabbe sattā sabba-dukkhā pamuccantu\\
Sabbe sattā laddha-sampattito mā vigacchantu

\bigskip

\begin{pali-hangtogether}
  Sabbe sattā kamma-ssakā kamma-dāy'ādā kamma-yonī kamma-bandhū kamma-paṭisaraṇā\\
\end{pali-hangtogether}
Yaṁ kammaṁ karissanti\\
Kalyāṇaṁ vā pāpakaṁ vā\\
Tassa dāy'ādā bhavissanti

\clearpage

\begin{leader-english}
  \anglebracketleft\ \hspace{-0.5mm}Now let us recite the reflections on universal well-being \hspace{-0.5mm}\anglebracketright\
\end{leader-english}

\begin{english-verses}
  May I abide in well-being\\
  In freedom from affliction\\
  In freedom from hostility\\
  In freedom from ill-will\\
  In freedom from anxiety\\
  And may I maintain well-being in myself\\
  May everyone abide in well-being\\
  In freedom from hostility\\
  In freedom from ill-will\\
  In freedom from anxiety\\
  And may they maintain well-being in themselves\\
  May all beings be released from all suffering\\
  \linkdest{endnote95-body}
  \begin{english-hangtogether}
    And may they not be parted from the good fortune they have attained\makeatletter\hyperlink{endnote95-appendix}\Hy@raisedlink{{\pagenote{%
          \hypertarget{endnote95-appendix}{\hyperlink{endnote95-body}{In the original version, this line is followed by ``When they act upon intention'', which is not found in the Pāli, and is potentially misleading, giving the implication that intention alone is not enough to count as kamma.}}}}}\makeatother
  \end{english-hangtogether}
\end{english-verses}

\linkdest{endnote96-body}
\begin{english-verses}
  All beings are the owners of their kamma\ifdigitalversion\makeatletter\hyperlink{endnote96-appendix}\Hy@raisedlink{{\pagenote{%
        \hypertarget{endnote96-appendix}{\hyperlink{endnote96-body}{WPN: ``All beings are the owners of their action and inherit its results. Their future is born from such action, companion to such action, and its results will be their home. All actions with intention, be they skilful or harmful, of such acts they will be the heirs''. For the sake of consistency with other chants within this chanting book, the original version was replaced with the one found in the ``Five Subjects for Frequent Reflection'', and ``Ten Subjects for Frequent Reflection''.}}}}}\makeatother\fi\\
  Heirs to their kamma\\
  Born of their kamma\\
  Related to their kamma\\
  Abide supported by their kamma\\
  Whatever kamma they shall do\\
  Either skillful or harmful\\
  Of such acts \breathmark\ they will be the heirs\\
\end{english-verses}

\suttaRef{[AN 3.65 \& AN 5.57]}

\bottomNav{seven-factors-of-awakening}



\sectionSubTitle{Brahmavihārā}
\section{The Divine Abidings}
\label{divine-abidings}

\begin{leader}
  \anglebracketleft\ \hspace{-0.5mm}Handa mayaṁ caturappamaññā obhāsanaṁ karomase \hspace{-0.5mm}\anglebracketright\
\end{leader}

\begin{pali-hang}
  Mettā-sahagatena cetasā ekaṁ disaṁ pharitvā viharati tathā dutiyaṁ tathā tatiyaṁ tathā catutthaṁ iti uddham'adho tiriyaṁ sabbadhi sabb'attatāya sabbāvantaṁ lokaṁ mettā-sahagatena cetasā vipulena mahaggatena appamāṇena averena abyāpajjhena pharitvā viharati
\end{pali-hang}

\medskip

\begin{pali-hang}
  Karuṇā-sahagatena cetasā ekaṁ disaṁ pharitvā viharati tathā dutiyaṁ tathā tatiyaṁ tathā catutthaṁ iti uddham'adho tiriyaṁ sabbadhi sabb'attatāya sabbāvantaṁ lokaṁ karuṇā-sahagatena cetasā vipulena mahaggatena appamāṇena averena abyāpajjhena pharitvā viharati
\end{pali-hang}

\medskip

\begin{pali-hang}
  Muditā-sahagatena cetasā ekaṁ disaṁ pharitvā viharati tathā dutiyaṁ tathā tatiyaṁ tathā catutthaṁ iti uddham'adho tiriyaṁ sabbadhi sabb'attatāya sabbāvantaṁ lokaṁ muditā-sahagatena cetasā vipulena mahaggatena appamāṇena averena abyāpajjhena pharitvā viharati
\end{pali-hang}

\medskip

\begin{pali-hang}
  Upekkhā-sahagatena cetasā ekaṁ disaṁ pharitvā viharati tathā dutiyaṁ tathā tatiyaṁ tathā catutthaṁ iti uddham'adho tiriyaṁ sabbadhi sabb'attatāya sabbāvantaṁ lokaṁ upekkhā-sahagatena cetasā vipulena mahaggatena appamāṇena averena abyāpajjhena pharitvā viharatī'ti
\end{pali-hang}

\clearpage

\begin{leader-english}
  \anglebracketleft\ \hspace{-0.5mm}Now let us make the Four Boundless Qualities shine forth \hspace{-0.5mm}\anglebracketright\
\end{leader-english}

\smallskip

\begin{english-hang}
  I will abide pervading one quarter with a heart imbued with loving-kindness
\end{english-hang}

\begin{english}
  Likewise the second likewise the third likewise the fourth\\
\end{english}
\begin{english-hang}
  So above and below \breathmark\ around and everywhere \breathmark\ and to all as to myself\makeatletter\hyperlink{endnote96-appendix}\Hy@raisedlink{{\pagenote{%
        \hypertarget{endnote96-appendix}{\hyperlink{endnote96-body}{The compound \textit{sabbattatāya} (\textit{sabba-t-tatāya}) seems to be an extension of its predecessor \textit{sabbadhi} (everywhere), and could thus even more suitably be translated as ``spreading to the entire (world)''; \textit{tata}: extended; spread out. (pp. of \textit{tanoti})}}}}}\makeatother\\
\end{english-hang}

\begin{english-hang}
  I will abide pervading the all-encompassing world \breathmark\ with a heart imbued with loving-kindness
\end{english-hang}

\begin{english-hang}
  Abundant exalted immeasurable \breathmark\ without hostility and without ill-will
\end{english-hang}

\medskip

\begin{english-hang}
  I will abide pervading one quarter with a heart imbued with compassion
\end{english-hang}

\begin{english}
  Likewise the second likewise the third likewise the fourth\\
  So above and below \breathmark\ around and everywhere \breathmark\ and to all as to myself
\end{english}

\begin{english-hang}
  I will abide pervading the all-encompassing world \breathmark\ with a heart imbued with compassion
\end{english-hang}

\begin{english-hang}
  Abundant exalted immeasurable \breathmark\ without hostility and without ill-will
\end{english-hang}

\medskip

\linkdest{endnote97-body}
\begin{english-hang}
  I will abide pervading one quarter with a heart imbued with empathetic joy\ifdigitalversion\makeatletter\hyperlink{endnote97-appendix}\Hy@raisedlink{{\pagenote{%
        \hypertarget{endnote97-appendix}{\hyperlink{endnote97-body}{WPN: ``a heart imbued with gladness''}}}}}\makeatother\fi
\end{english-hang}

\begin{english}
  Likewise the second likewise the third likewise the fourth\\
  So above and below \breathmark\ around and everywhere \breathmark\ and to all as to myself
\end{english}

\begin{english-hang}
  I will abide pervading the all-encompassing world \breathmark\ with a heart imbued with empathetic joy
\end{english-hang}

\begin{english-hang}
  Abundant exalted immeasurable \breathmark\ without hostility and without ill-will
\end{english-hang}

\smallskip

\begin{english-hang}
  I will abide pervading one quarter with a heart imbued with equanimity
\end{english-hang}

\begin{english}
  Likewise the second likewise the third likewise the fourth\\
  So above and below \breathmark\ around and everywhere \breathmark\ and to all as to myself
\end{english}

\begin{english-hang}
  I will abide pervading the all-encompassing world \breathmark\ with a heart imbued with equanimity
\end{english-hang}

\begin{english-hang}
  Abundant exalted immeasurable \breathmark\ without hostility and without ill-will
\end{english-hang}

\suttaRef{[DN 13]}

\bottomNav{ten-reflections}



\sectionSubTitle{Pañca-abhiṇha-paccavekkhaṇā}
\section{Five Subjects for Frequent Reflection}
\label{five-reflections}

\begin{leader}
  \anglebracketleft\ \hspace{-0.5mm}Handa mayaṁ abhiṇha-paccavekkhaṇa-pāṭhaṁ bhaṇāmase \hspace{-0.5mm}\anglebracketright\
\end{leader}

Jarā-dhammo'mhi jaraṁ anatīto

\begin{english}
  I am of the nature to age\\
  I have not gone beyond ageing
\end{english}

Byādhi-dhammo'mhi byādhiṁ anatīto

\begin{english}
  I am of the nature to sicken\\
  I have not gone beyond sickness
\end{english}

Maraṇa-dhammo'mhi maraṇaṁ anatīto

\begin{english}
  I am of the nature to die\\
  I have not gone beyond dying
\end{english}

Sabbehi me piyehi manāpehi nānābhāvo vinābhāvo

\begin{english}
  All that is mine beloved and pleasing\\
  Will become otherwise\\
  Will become separated from me
\end{english}

\begin{pali-hang}
  Kamma-ssako'mhi kamma-dāy'ādo kamma-yoni kamma-bandhu kamma-paṭisaraṇo\\
\end{pali-hang}
\begin{pali-hangtogether}
  Yaṁ kammaṁ karissāmi\\
\end{pali-hangtogether}
\begin{pali-hangtogether}
  Kalyāṇaṁ vā pāpakaṁ vā\\
\end{pali-hangtogether}
\begin{pali-hangtogether}
  Tassa dāy'ādo bhavissāmi
\end{pali-hangtogether}

\begin{english-verses}
  I am the owner of my kamma\\
  Heir to my kamma\\
  Born of my kamma\\
  Related to my kamma\\
  Abide supported by my kamma\\
  Whatever kamma I shall do\\
  Either skillful or harmful\\
  \linkdest{endnote98-body}
  Of such acts \breathmark\ I will be the heir\ifdigitalversion\makeatletter\hyperlink{endnote98-appendix}\Hy@raisedlink{{\pagenote{%
        \hypertarget{endnote98-appendix}{\hyperlink{endnote98-body}{WPN: `For good or for ill, of that I will be the heir' For the sake of consistency with other passages which were translated differently, that translation has been chosen.}}}}}\makeatother\fi
\end{english-verses}

Evaṁ amhehi abhiṇhaṁ paccavekkhitabbaṁ

\linkdest{endnote146-body}
\begin{english}
  Thus we should frequently reflect\ifdigitalversion\makeatletter\hyperlink{endnote146-appendix}\Hy@raisedlink{{\pagenote{%
        \hypertarget{endnote146-appendix}{\hyperlink{endnote146-body}{WPN: ``recollect''}}}}}\makeatother\fi
\end{english}

\suttaRef{[AN 5.57]}

\bottomNav{32-parts}



\sectionSubTitle{Dasadhammā pabbajita-abhiṇha-paccavekkhaṇā}
\section{Ten Subjects for Frequent Reflection by One Gone Forth}
\label{ten-reflections}

\begin{leader}
  \anglebracketleft\ \hspace{-0.5mm}Handa mayaṁ pabbajita-abhiṇha-paccavekkhaṇa-pāṭhaṁ bhaṇāmase \hspace{-0.5mm}\anglebracketright\
\end{leader}

Dasa ime bhikkhave dhammā\\
Pabbajitena abhiṇhaṁ paccavekkhitabbā\\
Katame dasa

\linkdest{endnote99-body}
\begin{english-hang-verses}
  Bhikkhus there are these ten dhammas\ifdigitalversion\makeatletter\hyperlink{endnote99-appendix}\Hy@raisedlink{{\pagenote{%
        \hypertarget{endnote99-appendix}{\hyperlink{endnote99-body}{WPN: ``\textit{Bhikkhus} there are ten \textit{dhammas}''}}}}}\makeatother\fi
  \breathmark\ which should be reflected upon again and again by one who has gone forth\\
\end{english-hang-verses}

\begin{english}
  What are these ten?
\end{english}

Vevaṇṇiy'amhi ajjhūpagato'ti\\
Pabbajitena abhiṇhaṁ paccavekkhitabbaṁ

\linkdest{endnote100-body}
\begin{english-verses}
  ``I have reached a state of castelessness''\ifdigitalversion\makeatletter\hyperlink{endnote100-appendix}\Hy@raisedlink{{\pagenote{%
        \hypertarget{endnote100-appendix}{\hyperlink{endnote100-body}{WPN: ``I am no longer living according to worldly aims and values''}}}}}\makeatother\fi
  \begin{english-hangtogether-verses}
    This should be reflected upon again and again by one who has gone forth
  \end{english-hangtogether-verses}
\end{english-verses}

Para-paṭibaddhā me jīvikā'ti\\
Pabbajitena abhiṇhaṁ paccavekkhitabbaṁ

\begin{english-verses}
  ``My very life is sustained through the gifts of others''
  \begin{english-hangtogether-verses}
    This should be reflected upon again and again by one who has gone forth
  \end{english-hangtogether-verses}
\end{english-verses}

\ifasixversion\clearpage\fi

Añño me ākappo karaṇīyo'ti\\
Pabbajitena abhiṇhaṁ paccavekkhitabbaṁ

\ifafiveversion\clearpage\fi

\linkdest{endnote101-body}
\begin{english-verses}
  ``Now my conduct should be different from before''\ifdigitalversion\makeatletter\hyperlink{endnote101-appendix}\Hy@raisedlink{{\pagenote{%
        \hypertarget{endnote101-appendix}{\hyperlink{endnote101-body}{WPN: ``I should strive to abandon my former habits''}}}}}\makeatother\fi
  \begin{english-hangtogether-verses}
    This should be reflected upon again and again by one who has gone forth
  \end{english-hangtogether-verses}
\end{english-verses}

Kacci nu kho me attā sīlato na upavadatī'ti\\
Pabbajitena abhiṇhaṁ paccavekkhitabbaṁ

\begin{english-verses}
  ``Does regret over my conduct arise in my mind?''
  \begin{english-hangtogether-verses}
    This should be reflected upon again and again by one who has gone forth
  \end{english-hangtogether-verses}
\end{english-verses}

\begin{pali-hang}
  Kacci nu kho maṁ anuvicca viññū sabrahmacārī sīlato na upavadantī'ti
\end{pali-hang}
\begin{pali-hangtogether}
  Pabbajitena abhiṇhaṁ paccavekkhitabbaṁ
\end{pali-hangtogether}

\begin{english-verses}
  ``Could my spiritual companions find fault with my conduct?''
  \begin{english-hangtogether-verses}
    This should be reflected upon again and again by one who has gone forth
  \end{english-hangtogether-verses}
\end{english-verses}

Sabbehi me piyehi manāpehi nānābhāvo vinābhāvo'ti\\
Pabbajitena abhiṇhaṁ paccavekkhitabbaṁ

\begin{english-verses}
  ``All that is mine beloved and pleasing\\
  Will become otherwise\\
  Will become separated from me''
  \begin{english-hangtogether-verses}
    This should be reflected upon again and again by one who has gone forth
  \end{english-hangtogether-verses}
\end{english-verses}

\ifasixversion\clearpage\fi

\begin{pali-hang}
  Kamma-ssako'mhi kamma-dāy'ādo kamma-yoni kamma-bandhu kamma-paṭisaraṇo
\end{pali-hang}
\begin{pali-hangtogether}
  Yaṁ kammaṁ karissāmi
\end{pali-hangtogether}
\begin{pali-hangtogether}
  Kalyāṇaṁ vā pāpakaṁ vā
\end{pali-hangtogether}
\begin{pali-hangtogether}
  Tassa dāy'ādo bhavissāmī'ti
\end{pali-hangtogether}
\begin{pali-hangtogether}
  Pabbajitena abhiṇhaṁ paccavekkhitabbaṁ
\end{pali-hangtogether}

\begin{english-verses}
  ``I am the owner of my kamma\\
  Heir to my kamma\\
  Born of my kamma\\
  Related to my kamma\\
  Abide supported by my kamma\\
  Whatever kamma I shall do\\
  Either skillful or harmful\\
  \linkdest{endnote102-body}
  Of such acts \breathmark\ I will be the heir''\ifdigitalversion\makeatletter\hyperlink{endnote102-appendix}\Hy@raisedlink{{\pagenote{%
        \hypertarget{endnote102-appendix}{\hyperlink{endnote102-body}{WPN: ``For good or for ill, Of that I will be the heir'' For the sake of consistency with other passages which were translated differently, this translation has been chosen.}}}}}\makeatother\fi
  \begin{english-hangtogether-verses}
    This should be reflected upon again and again by one who has gone forth
  \end{english-hangtogether-verses}
\end{english-verses}

`Katham'bhūtassa me rattin'divā vītipatantī'ti\\
Pabbajitena abhiṇhaṁ paccavekkhitabbaṁ

\begin{english-verses}
  ``The days and nights are relentlessly passing\\
  How well am I spending my time?''
  \begin{english-hangtogether-verses}
    This should be reflected upon again and again by one who has gone forth
  \end{english-hangtogether-verses}
\end{english-verses}

Kacci nu kho'haṁ suññ'āgāre abhiramāmī'ti\\
Pabbajitena abhiṇhaṁ paccavekkhitabbaṁ

\begin{english-verses}
  ``Do I delight in solitude or not?''
  \begin{english-hangtogether-verses}
    This should be reflected upon again and again by one who has gone forth
  \end{english-hangtogether-verses}
\end{english-verses}

\begin{pali-hang}
  Atthi nu kho me uttari-manussa-dhammā alam'ariya-ñāṇa-dassana-viseso adhigato
\end{pali-hang}
\begin{pali-hangtogether}
  So'haṁ pacchime kāle sabrahmacārīhi puṭṭho na maṅku bhavissāmī'ti
\end{pali-hangtogether}
\begin{pali-hangtogether}
  Pabbajitena abhiṇhaṁ paccavekkhitabbaṁ
\end{pali-hangtogether}

\begin{english-verses}
  ``Has my practice borne fruit with freedom or insight\\
  \begin{english-hangtogether-verses}
    So that at the end of my life \breathmark\ I need not feel ashamed when questioned by my spiritual companions?''
  \end{english-hangtogether-verses}
  \begin{english-hangtogether-verses}
    This should be reflected upon again and again by one who has gone forth
  \end{english-hangtogether-verses}
\end{english-verses}

Ime kho bhikkhave dasa dhammā\\
Pabbajitena abhiṇhaṁ paccavekkhitabbā'ti

\begin{english-hang-verses}
  Bhikkhus these are the ten dhammas \breathmark\ which should be reflected upon again and again by one who has gone forth
\end{english-hang-verses}

\suttaRef{[AN 10.48]}

\bottomNav{sharing-aspirations}



\sectionSubTitle{Dvattiṁs'ākāra-paccavekkhaṇa}
\section{The Thirty-Two Body Parts}
\label{32-parts}

\begin{leader}
  \anglebracketleft\ \hspace{-0.5mm}Handa mayaṁ dvattiṁs'ākāra-pāṭhaṁ bhaṇāmase \hspace{-0.5mm}\anglebracketright\
\end{leader}

\begin{pali-hang}
  Ayaṁ kho me kāyo uddhaṁ pādatalā adho kesa-matthakā taca-pariyanto pūro nāna-ppakārassa asucino
\end{pali-hang}

\begin{english-verses}
  This which is my body\\
  From the soles of the feet up\\
  And down from the crown of the head\\
  Is a sealed bag of skin\\
  Filled with unattractive things
\end{english-verses}

Atthi imasmiṁ kāye

\begin{english}
  In this body there are
\end{english}

\ifafiveversion

{\centering
  \setArrayStretch{1}

  \begin{tabular}{ r l }
    kesā            & \tr{hair of the head} \\
    lomā            & \tr{hair of the body} \\
    nakhā           & \tr{nails} \\
    dantā           & \tr{teeth} \\
    taco            & \tr{skin} \\
    maṁsaṁ          & \tr{flesh}\\
    nahārū          & \tr{sinews} \\
    aṭṭhī           & \tr{bones} \\
    aṭṭhimiñjaṁ     & \tr{bone marrow} \\
    vakkaṁ          & \tr{kidneys} \\
    hadayaṁ         & \tr{heart} \\
    yakanaṁ         & \tr{liver} \\
    kilomakaṁ       & \tr{membranes} \\
  \end{tabular}
  \linkdest{endnote152-body}
  \linkdest{endnote153-body}
  \begin{tabular}{ r l }
    pihakaṁ         & \tr{spleen} \\
    papphāsaṁ       & \tr{lungs} \\
    antaṁ           & \tr{intestines}\ifdigitalversion\makeatletter\hyperlink{endnote152-appendix}\Hy@raisedlink{{\pagenote{%
                      \hypertarget{endnote152-appendix}{\hyperlink{endnote152-body}{WPN: ``bowels''}}}}}\fi \\
    antaguṇaṁ       & \tr{mesentery}\ifdigitalversion\makeatletter\hyperlink{endnote153-appendix}\Hy@raisedlink{{\pagenote{%
                      \hypertarget{endnote153-appendix}{\hyperlink{endnote153-body}{WPN: ``entrails''}}}}}\fi \\
    udariyaṁ        & \tr{undigested food} \\
    karīsaṁ         & \tr{excrement} \\
    pittaṁ          & \tr{bile} \\
    semhaṁ          & \tr{phlegm} \\
    pubbo           & \tr{pus} \\
    lohitaṁ         & \tr{blood} \\
    sedo            & \tr{sweat} \\
    medo            & \tr{fat} \\
    assu            & \tr{tears} \\
    vasā            & \tr{grease} \\
    kheḷo           & \tr{spittle} \\
    siṅghāṇikā      & \tr{mucus} \\
    lasikā          & \tr{oil of the joints} \\
    muttaṁ          & \tr{urine} \\
    \linkdest{endnote103-body}
    matthaluṅgan'ti & \tr{brain}\makeatletter\hyperlink{endnote103-appendix}\Hy@raisedlink{{\pagenote{%
                      \hypertarget{endnote103-appendix}{\hyperlink{endnote103-body}{In the discourses, except for one occasion in the Khp, the brain is not mentioned as a separate organ or body part, making it a list of only 31 body parts.}}}}}\makeatother
  \end{tabular}

  \restoreArrayStretch
}

\fi

\ifasixversion

{\centering
  \setArrayStretch{1}

  \begin{tabular}{ r l }
    kesā            & \tr{hair of the head} \\
    lomā            & \tr{hair of the body} \\
    nakhā           & \tr{nails} \\
    dantā           & \tr{teeth} \\
    taco            & \tr{skin} \\
    maṁsaṁ          & \tr{flesh}\\
    nahārū          & \tr{sinews} \\
    aṭṭhī           & \tr{bones} \\
    aṭṭhimiñjaṁ     & \tr{bone marrow} \\
    vakkaṁ          & \tr{kidneys} \\
    hadayaṁ         & \tr{heart} \\
  \end{tabular}
  \linkdest{endnote152-body}
  \linkdest{endnote153-body}
  \begin{tabular}{ r l }
    yakanaṁ         & \tr{liver} \\
    kilomakaṁ       & \tr{membranes} \\
    pihakaṁ         & \tr{spleen} \\
    papphāsaṁ       & \tr{lungs} \\
    antaṁ           & \tr{intestines}\ifdigitalversion\makeatletter\hyperlink{endnote152-appendix}\Hy@raisedlink{{\pagenote{%
                      \hypertarget{endnote152-appendix}{\hyperlink{endnote152-body}{WPN: ``bowels''}}}}}\fi \\
    antaguṇaṁ       & \tr{mesentery}\ifdigitalversion\makeatletter\hyperlink{endnote153-appendix}\Hy@raisedlink{{\pagenote{%
                      \hypertarget{endnote153-appendix}{\hyperlink{endnote153-body}{WPN: ``entrails''}}}}}\fi \\
    udariyaṁ        & \tr{undigested food} \\
    karīsaṁ         & \tr{excrement} \\
    pittaṁ          & \tr{bile} \\
    semhaṁ          & \tr{phlegm} \\
    pubbo           & \tr{pus} \\
    lohitaṁ         & \tr{blood} \\
    sedo            & \tr{sweat} \\
    medo            & \tr{fat} \\
    assu            & \tr{tears} \\
    vasā            & \tr{grease} \\
    kheḷo           & \tr{spittle} \\
    siṅghāṇikā      & \tr{mucus} \\
    lasikā          & \tr{oil of the joints} \\
    muttaṁ          & \tr{urine} \\
    \linkdest{endnote103-body}
    matthaluṅgan'ti & \tr{brain}\makeatletter\hyperlink{endnote103-appendix}\Hy@raisedlink{{\pagenote{%
                      \hypertarget{endnote103-appendix}{\hyperlink{endnote103-body}{In the discourses, except for one occasion in the Khp, the brain is not mentioned as a separate organ or body part, making it a list of only 31 body parts.}}}}}\makeatother
  \end{tabular}

  \restoreArrayStretch
}

\fi

\ifbsixversion

{\centering
  \setArrayStretch{1}

  \begin{tabular}{ r l }
    kesā            & \tr{hair of the head} \\
    lomā            & \tr{hair of the body} \\
    nakhā           & \tr{nails} \\
    dantā           & \tr{teeth} \\
    taco            & \tr{skin} \\
    maṁsaṁ          & \tr{flesh}\\
    nahārū          & \tr{sinews} \\
    aṭṭhī           & \tr{bones} \\
    aṭṭhimiñjaṁ     & \tr{bone marrow} \\
    vakkaṁ          & \tr{kidneys} \\
    hadayaṁ         & \tr{heart} \\
    yakanaṁ         & \tr{liver} \\
    kilomakaṁ       & \tr{membranes} \\
  \end{tabular}
  \linkdest{endnote152-body}
  \linkdest{endnote153-body}
  \begin{tabular}{ r l }
    pihakaṁ         & \tr{spleen} \\
    papphāsaṁ       & \tr{lungs} \\
    antaṁ           & \tr{intestines}\ifdigitalversion\makeatletter\hyperlink{endnote152-appendix}\Hy@raisedlink{{\pagenote{%
                      \hypertarget{endnote152-appendix}{\hyperlink{endnote152-body}{WPN: ``bowels''}}}}}\fi \\
    antaguṇaṁ       & \tr{mesentery}\ifdigitalversion\makeatletter\hyperlink{endnote153-appendix}\Hy@raisedlink{{\pagenote{%
                      \hypertarget{endnote153-appendix}{\hyperlink{endnote153-body}{WPN: ``entrails''}}}}}\fi \\
    udariyaṁ        & \tr{undigested food} \\
    karīsaṁ         & \tr{excrement} \\
    pittaṁ          & \tr{bile} \\
    semhaṁ          & \tr{phlegm} \\
    pubbo           & \tr{pus} \\
    lohitaṁ         & \tr{blood} \\
    sedo            & \tr{sweat} \\
    medo            & \tr{fat} \\
    assu            & \tr{tears} \\
    vasā            & \tr{grease} \\
    kheḷo           & \tr{spittle} \\
    siṅghāṇikā      & \tr{mucus} \\
    lasikā          & \tr{oil of the joints} \\
    muttaṁ          & \tr{urine} \\
    \linkdest{endnote103-body}
    matthaluṅgan'ti & \tr{brain}\makeatletter\hyperlink{endnote103-appendix}\Hy@raisedlink{{\pagenote{%
                      \hypertarget{endnote103-appendix}{\hyperlink{endnote103-body}{In the discourses, except for one occasion in the Khp, the brain is not mentioned as a separate organ or body part, making it a list of only 31 body parts.}}}}}\makeatother
  \end{tabular}

  \restoreArrayStretch
}

\fi

\begin{pali-hang}
  Evam'ayaṁ me kāyo uddhaṁ pādatalā adho kesa-matthakā taca-pariyanto pūro nāna-ppakārassa asucino
\end{pali-hang}

\begin{english-verses}
  This then which is my body\\
  From the soles of the feet up\\
  And down from the crown of the head\\
  Is a sealed bag of skin\\
  Filled with unattractive things
\end{english-verses}

\suttaRef{[DN 22]}

\enlargethispage{\baselineskip\vspace{-1.0em}}
\bottomNav{principles-of-cordiality}



\sectionSubTitle{Anicc'ānussati}
\section{Recollection of Impermanence}
\label{recollection-of-impermanence}

\begin{leader}
  \anglebracketleft\ \hspace{-0.5mm}Handa mayaṁ anicc'ānussati-pāṭhaṁ bhaṇāmase \hspace{-0.5mm}\anglebracketright\
\end{leader}

Sabbe saṅkhārā aniccā

\begin{english}
  All conditioned things are impermanent
\end{english}

Sabbe saṅkhārā dukkhā

\begin{english}
  All conditioned things are dukkha
\end{english}

Sabbe dhammā anattā

\linkdest{endnote104-body}
\begin{english}
  All things are not-self\ifdigitalversion\makeatletter\hyperlink{endnote104-appendix}\Hy@raisedlink{{\pagenote{%
        \hypertarget{endnote104-appendix}{\hyperlink{endnote104-body}{WPN: ``Everything is void of self''}}}}}\makeatother\fi
\end{english}

\suttaRef{[Dhp 277-279]}

Addhuvaṁ jīvitaṁ

\begin{english}
  Life is not for sure
\end{english}

Dhuvaṁ maraṇaṁ

\begin{english}
  Death is for sure
\end{english}

Avassaṁ mayā maritabbaṁ

\begin{english}
  It is inevitable that I'll die
\end{english}

Maraṇa-pariyosānaṁ me jīvitaṁ

\begin{english}
  Death is the culmination of my life
\end{english}

Jīvitaṁ me aniyataṁ

\begin{english}
  My life is uncertain
\end{english}

\ifasixversion\clearpage\fi

Maraṇaṁ me niyataṁ

\begin{english}
  My death is certain
\end{english}

\suttaRef{[Dhp A]}

\ifafiveversion\clearpage\fi

Vata

\begin{english}
  Indeed
\end{english}

Ayaṁ kāyo

\begin{english}
  This body
\end{english}

Aciraṁ

\begin{english}
  Will soon
\end{english}

Apeta-viññāṇo

\begin{english}
  Be void of consciousness
\end{english}

Chuḍḍho

\begin{english}
  And cast away
\end{english}

Adhisessati

\begin{english}
  It will lie
\end{english}

Paṭhaviṁ

\begin{english}
  On the ground
\end{english}

Kaliṅgaraṁ iva

\begin{english}
  Just like a rotten log
\end{english}

Niratthaṁ

\linkdest{endnote105-body}
\begin{english}
  Useless\ifdigitalversion\makeatletter\hyperlink{endnote105-appendix}\Hy@raisedlink{{\pagenote{%
        \hypertarget{endnote105-appendix}{\hyperlink{endnote105-body}{WPN: ``Completely void of use''}}}}}\makeatother\fi
\end{english}

\suttaRef{[Dhp 41]}

Aniccā vata saṅkhārā

\begin{english}
  Indeed conditioned things cannot last
\end{english}

Uppāda-vaya-dhammino

\linkdest{endnote106-body}
\begin{english}
  Their nature is to rise and cease\ifdigitalversion\makeatletter\hyperlink{endnote106-appendix}\Hy@raisedlink{{\pagenote{%
        \hypertarget{endnote106-appendix}{\hyperlink{endnote106-body}{WPN: ``Their nature is to rise and fall''}}}}}\makeatother\fi
\end{english}

Uppajjitvā nirujjhanti

\begin{english}
  Having arisen things must cease
\end{english}

Tesaṁ vūpasamo sukho

\begin{english}
  Their stilling is true happiness
\end{english}

\suttaRef{[Trad]}

\bottomNav{yatha-vari-vaha-pura}

