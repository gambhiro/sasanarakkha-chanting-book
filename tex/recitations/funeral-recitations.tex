
\ifdigitalversion
  \chapterOpeningPage{funeral-recitations-compressed.jpg}
\else
  \chapterOpeningPage{funeral-recitations.jpg}
\fi

\chapter{Funeral Recitations}

\clearpage



\sectionSubTitle{Just as Rivers}
\section{Pubba-bhāga-nama-kāra-pāṭho}
\label{pubba-bhaga-nama-kara-patho-funeral}


\begin{pali-hangtogether}
  \anglebracketleft\ \hspace{-0.5mm}Namo tassa \hspace{-0.5mm}\anglebracketright\ bhagavato arahato sammāsambuddhassa \hfill{[3x]}
\end{pali-hangtogether}

\begin{english}
  Homage to the Blessed, Worthy, and Perfectly Enlightened One
\end{english}

\suttaRef{[DN 21]}



\sectionSubTitle{Just as Rivers}
\section{Dhamma-saṅgaṇī-mātikā}
\label{dhamma-sangani-matika}

\begin{pali-hangtogether}
  \anglebracketleft\ \hspace{-0.5mm}Kusalā dhammā \hspace{-0.5mm}\anglebracketright\ \\
  Akusalā dhammā\\
  Abyākatā dhammā
\end{pali-hangtogether}

\begin{english-verses}
  Wholesome dhammas\\
  Unwholesome dhammas\\
  Undetermined dhammas
\end{english-verses}

\begin{pali-hang-continued}
  Sukhāya vedanāya sampayuttā dhammā\\
  Dukkhāya vedanāya sampayuttā dhammā\\
  Adukkham'asukhāya vedanāya sampayuttā dhammā
\end{pali-hang-continued}

\begin{english-verses}
  Dhammas associated with pleasant feeling\\
  Dhammas associated with painful feeling\\
  Dhammas associated with neither-painful-nor-pleasant feeling
\end{english-verses}

\begin{pali-hang-continued}
  Vipākā dhammā\\
  Vipāka-dhamma-dhammā\\
  N'eva vipāka na vipāka-dhamma-dhammā
\end{pali-hang-continued}

\begin{english-verses}
  Consequential dhamma\\
  Subject to consequential dhamma\\
  Neither consequential nor subject to consequential dhamma
\end{english-verses}

\begin{pali-hang-continued}
  Upādinn'upādāniyā dhammā\\
  Anupādinn'upādāniyā dhammā\\
  Anupādinn'ānupādāniyā dhammā
\end{pali-hang-continued}

\begin{english-verses}
  Clung dhammas which can be clung to\\
  Unclung dhammas which can be clung to\\
  Unclung dhammas which cannot be clung to
\end{english-verses}

\begin{pali-hang-continued}
  Saṅkiliṭṭha-saṅkilesikā dhammā\\
  Asaṅkiliṭṭha-saṅkilesikā dhammā\\
  Asaṅkiliṭṭh'āsaṅkilesikā dhammā
\end{pali-hang-continued}

\begin{english-verses}
  Dhammas defiled and subject to defilements\\
  Dhammas undefiled but subject to defilements\\
  Dhammas neither defiled nor subject to defilements
\end{english-verses}

\begin{pali-hang-continued}
  Savitakka-savicārā dhammā\\
  Avitakka-vicāra-mattā dhammā\\
  Avitakk'āvicārā dhammā
\end{pali-hang-continued}

\begin{english-verses}
  Dhammas with thought and examination\\
  Dhammas without thought but with examination\\
  Dhammas with neither thought nor examination
\end{english-verses}

\begin{pali-hang-continued}
  Pīti-sahagatā dhammā\\
  Sukha-sahagatā dhammā\\
  Upekkhā-sahagatā dhammā
\end{pali-hang-continued}

\begin{english-verses}
  Dhammas accompanied by rapture\\
  Dhammas accompanied by pleasure\\
  Dhammas accompanied by equanimity
\end{english-verses}

\begin{pali-hang-continued}
  Dassanena pahātabbā dhammā\\
  Bhāvanāya pahātabbā dhammā\\
  N'eva dassanena na bhāvanāya pahātabbā dhammā
\end{pali-hang-continued}

\begin{english-verses}
  Dhammas abandoned by seeing\\
  Dhammas abandoned by development\\
  Dhammas abandoned by neither seeing nor development
\end{english-verses}

\begin{pali-hang-continued}
  Dassanena pahātabba-hetukā dhammā\\
  Bhāvanāya pahātabba-hetukā dhammā\\
  N'eva dassanena na bhāvanāya pahātabba-hetukā dhammā
\end{pali-hang-continued}

\begin{english-verses}
  Conditioned dhammas abandoned by seeing\\
  Conditioned dhammas abandoned by development\\
  Conditioned dhammas abandoned by neither seeing nor development
\end{english-verses}

\begin{pali-hang-continued}
  Ācaya-gāmino dhammā\\
  Apacaya-gāmino dhammā\\
  N'ev'ācaya-gāmino n'āpacaya-gāmino dhammā
\end{pali-hang-continued}

\begin{english-verses}
  Dhammas leading to building up\\
  Dhamma leading to dismantling\\
  Dhammas leading to neither building up nor dismantling
\end{english-verses}

\begin{pali-hang-continued}
  Sekkhā dhammā\\
  Asekkhā dhammā\\
  N'eva sekkhā n'āsekkhā dhammā
\end{pali-hang-continued}

\begin{english-verses}
  Dhammas of one in training\\
  Dhammas of one beyond training\\
  Dhammas of neither one in training nor one beyond training
\end{english-verses}

\begin{pali-hang-continued}
  Parittā dhammā\\
  Mahaggatā dhammā\\
  Appamāṇā dhammā
\end{pali-hang-continued}

\begin{english-verses}
  Limited dhammas\\
  Exhalted dhammas\\
  Immeasurable dhammas
\end{english-verses}

\begin{pali-hang-continued}
  Paritt'ārammaṇā dhammā\\
  Mahaggat'ārammaṇā dhammā\\
  Appamāṇ'ārammaṇā dhammā
\end{pali-hang-continued}

\begin{english-verses}
  Dhammas from limited sense-obejcts\\
  Dhammas from exhalted sense-objects\\
  Dhammas from immeasurable sense-objects
\end{english-verses}

\begin{pali-hang-continued}
  Hīnā dhammā\\
  Majjhimā dhammā\\
  Paṇītā dhammā
\end{pali-hang-continued}

\begin{english-verses}
  Inferior dhammas\\
  Moderate dhammas\\
  Superior dhammas
\end{english-verses}

\begin{pali-hang-continued}
  Micchatta-niyatā dhammā\\
  Sammatta-niyatā dhammā\\
  Aniyatā dhammā
\end{pali-hang-continued}

\begin{english-verses}
  Certain wrong dhammas\\
  Certain correct dhammas\\
  Uncertain dhammas
\end{english-verses}

\begin{pali-hang-continued}
  Magg'ārammaṇā dhammā\\
  Magga-hetukā dhammā\\
  Magg'ādhipatino dhammā
\end{pali-hang-continued}

\begin{english-verses}
  Dhammas with the path as object\\
  Dhammas with the path as cause\\
  Dhammas with the path as predominant factor
\end{english-verses}

\begin{pali-hang-continued}
  Uppannā dhammā\\
  Anuppannā dhammā\\
  Uppādino dhammā
\end{pali-hang-continued}

\begin{english-verses}
  Arisen dhammas\\
  Unarisen dhammas\\
  Bound to arise dhammas
\end{english-verses}

\begin{pali-hang-continued}
  Atītā dhammā\\
  Anāgatā dhammā\\
  Paccuppannā dhammā
\end{pali-hang-continued}

\begin{english-verses}
  Past dhammas\\
  Future dhammas\\
  Present dhammas
\end{english-verses}

\begin{pali-hang-continued}
  Atīt'ārammaṇā dhammā\\
  Anāgat'ārammaṇā dhammā\\
  Paccuppann'ārammaṇā dhammā
\end{pali-hang-continued}

\begin{english-verses}
  Dhammas with past sense-objects\\
  Dhammas with future sense-objects\\
  Dhammas with present sense-objects
\end{english-verses}

\begin{pali-hang-continued}
  Ajjhattā dhammā\\
  Bahiddhā dhammā\\
  Ajjhatta-bahiddhā dhammā
\end{pali-hang-continued}

\begin{english-verses}
  Internal dhammas\\
  External dhammas\\
  Internal and external dhamams
\end{english-verses}

\begin{pali-hang-continued}
  Ajjhatt'ārammaṇā dhammā\\
  Bahiddh'ārammaṇā dhammā\\
  Ajjhatta-bahiddh'ārammaṇā dhammā
\end{pali-hang-continued}

\begin{english-verses}
  Dhammas with internal sense-objects\\
  Dhammas with external sense-objects\\
  Dhammas with internal and external sense-objects
\end{english-verses}

\begin{pali-hang-continued}
  Sanidassana-sappaṭighā dhammā\\
  Anidassana-sappaṭighā dhammā\\
  Anidassan'āppaṭighā dhammā
\end{pali-hang-continued}

\begin{english-verses}
  Visible and impinging dhammas\\
  Non-visible and impinging dhammas\\
  Non-visible and unimpinging dhammas
\end{english-verses}

\suttaRef{[Dhs 1]}



\sectionSubTitle{Just as Rivers}
\section{Vipassanā-bhūmi-pāṭho}
\label{vipassana-bhumi-patho}

% TODO fix hang indent and hang indent continued
% TODO don't let line end with the
\begin{pali-hang-continued}
  \anglebracketleft\ \hspace{-0.5mm}Pañca-kkhandhā: \hspace{-0.5mm}\anglebracketright\
  \begin{pali-hang}
    Rūpa-kkhandho vedanā-kkhandho saññā-kkhandho saṅkhāra-kkhandho viññāṇa-kkhandho
  \end{pali-hang}
\end{pali-hang-continued}

\begin{english-verses}
  Five aggregates:
  \begin{english-hangtogether-verses}
    The aggregate of form, the aggregate of feeling, the aggregate of perception, the aggregate of volitional formations, the aggregate of consciousness.
  \end{english-hangtogether-verses}
\end{english-verses}

\suttaRef{[MN 109]}

\begin{pali-hang-continued}
  Dvā-das'āyatanāni:
  \begin{pali-hang}
    Cakkhv'āyatanaṁ rūp'āyatanaṁ sot'āyatanaṁ sadd'āyatanaṁ ghān'āyatanaṁ gandh'āyatanaṁ jivh'āyatanaṁ ras'āyatanaṁ kāy'āyatanaṁ phoṭṭhabb'āyatanaṁ man'āyatanaṁ dhamm'āyatanaṁ
  \end{pali-hang}
\end{pali-hang-continued}

\begin{english-verses}
  Twelve sense bases:
  \begin{english-hangtogether-verses}
    The eye-base, the form base, the ear-base, the sound-base, the~nose-base, the odour-base, the tongue-base, the flavour-base, the~body-base, the tangible-base, the mind-base, the mind-object base.
  \end{english-hangtogether-verses}
\end{english-verses}

\suttaRef{[MN 148]}

\begin{pali-hang-continued}
  Aṭṭhārasa dhātuyo:
  \begin{pali-hang}
    Cakkhu-dhātu rūpa-dhātu cakkhu-viññāṇa-dhātu sota-dhātu sadda-dhātu sota-viññāṇa-dhātu ghāna-dhātu gandha-dhātu ghāna-viññāṇa-dhātu jivhā-dhātu rasa-dhātu jivhā-viññāṇa-dhātu kāya-dhātu phoṭṭhabba-dhātu kāya-viññāṇa-dhātu mano-dhātu dhamma-dhātu mano-viññāṇa-dhātu
  \end{pali-hang}
\end{pali-hang-continued}

\begin{english-verses}
  Eighteen elements:
  \begin{english-hangtogether-verses}
    The eye element, the form element, the eye-consciousness element; The ear element, the sound element, the ear-consciousness element; The nose element, the odour element, the nose-consciousness element; The tongue element, the flavour element, the tongue-consciousness element; The body element, the tangible element, the body-consciousness element; The mind element, the mind-object element, the mind-consciousness element.
  \end{english-hangtogether-verses}
\end{english-verses}

\suttaRef{[MN 115]}

\linkdest{endnote131-body}
\begin{pali-hang-continued}
  Bā-vīsat'indriyāni:\makeatletter\hyperlink{endnote131-appendix}\Hy@raisedlink{{\pagenote{%
        \hypertarget{endnote131-appendix}{\hyperlink{endnote131-body}{While these faculties are mentioned as a set of 22 only in the Abhi. Vibh., all of them are also found within the discourses.}}}}}\makeatother
  \begin{pali-hang}
    Cakkhu'ndriyaṁ sot'indriyaṁ ghān'indriyaṁ jivh'indriyaṁ kāy'indriyaṁ man'indriyaṁ itth'indriyaṁ puris'indriyaṁ jīvit'indriyaṁ sukh'indriyaṁ dukkh'indriyaṁ somanass'indriyaṁ domanass'indriyaṁ upekkh'indriyaṁ saddh'indriyaṁ viriy'indriyaṁ sat'indriyaṁ samādh'indriyaṁ paññ'indriyaṁ anaññātañ'ñassāmī't'indriyaṁ aññ'indriyaṁ aññātāv'indriyaṁ
  \end{pali-hang}
\end{pali-hang-continued}

\begin{english-verses}
  Twenty-two faculties:
  \begin{english-hangtogether-verses}
    The eye faculty, ear faculty, nose faculty, tongue faculty, body faculty, mind faculty, faculty of feminity, faculty of masculinity, life faculty, pleasure faculty, pain faculty, happiness faculty, displeasure faculty, equanimity faculty, conviction faculty, energy faculty, mindfulness faculty, concentration faculty, wisdom faculty, the `I am knowing the unknown' faculty, knowledge faculty, the faculty of one with complete knowledge.
  \end{english-hangtogether-verses}
\end{english-verses}

\suttaRef{[Vibh]}

\begin{pali-hang-continued}
  Cattāri ariya-saccāni:\\
  Dukkhaṁ ariya-saccaṁ\\
  Dukkha-samudayo ariya-saccaṁ\\
  Dukkha-nirodho ariya-saccaṁ\\
  Dukkha-nirodha-gāminī paṭipadā ariya-saccaṁ
\end{pali-hang-continued}

\begin{english-verses}
  Four noble truths:\\
  The noble truth of dukkha;\\
  The noble truth of the origin of dukkha;\\
  The noble truth of the cessation of dukkha;\\
  The noble truth of the way leading to the cessation of dukkha.
\end{english-verses}

\suttaRef{[SN 56.24]}

\begin{pali-hang-continued}
  Avijjā-paccayā saṅkhārā\\
  Saṅkhāra-paccayā viññāṇaṁ\\
  Viññāṇa-paccayā nāma-rūpaṁ\\
  Nāma-rūpa-paccayā saḷ'āyatanaṁ\\
  Saḷ'āyatana-paccayā phasso\\
  Phassa-paccayā vedanā\\
  Vedanā-paccayā taṇhā\\
  Taṇhā-paccayā upādānaṁ\\
  Upādāna-paccayā bhavo\\
  Bhava-paccayā jāti\\
  \begin{pali-hangtogether}
    Jāti-paccayā jarā-maraṇaṁ soka-parideva-dukkha-domanass'upāyāsā sambhavanti
  \end{pali-hangtogether}
  \begin{pali-hangtogether}
    Evam'etassa kevalassa dukkha-kkhandhassa samudayo hoti
  \end{pali-hangtogether}
\end{pali-hang-continued}

\begin{english-verses}
  With ignorance as condition, volitional formations;\\
  With volitional formations as condition, consciousness;\\
  With consciousness as condition, name-and-form;\\
  With name-and-form as condition, the six sense bases;\\
  With the six sense bases as condition, contact;\\
  With contact as condition, feeling;\\
  With feeling as condition, craving;\\
  With craving as condition, clinging;\\
  With clinging as condition, existence;\\
  With existence as condition, birth;
  \begin{english-hangtogether-verses}
    With birth as condition, ageing-and-death, sorrow, lamentation, pain, displeasure, and despair come to be.
  \end{english-hangtogether-verses}
  \begin{english-hangtogether-verses}
    Such is the origin of this whole mass of suffering.
  \end{english-hangtogether-verses}
\end{english-verses}

\begin{pali-hang-continued}
  Avijjāya tv'eva asesa-virāga-nirodhā saṅkhāra-nirodho\\
  Saṅkhāra-nirodhā viññāṇa-nirodho\\
  Viññāṇa-nirodhā nāma-rūpa-nirodho\\
  Nāma-rūpa-nirodhā saḷ'āyatana-nirodho\\
  Saḷ'āyatana-nirodhā phassa-nirodho\\
  Phassa-nirodhā vedanā-nirodho\\
  Vedanā-nirodhā taṇhā-nirodho\\
  Taṇhā-nirodhā upādāna-nirodho\\
  Upādāna-nirodhā bhava-nirodho\\
  Bhava-nirodhā jāti-nirodho
  \begin{pali-hangtogether}
    Jāti-nirodhā jarā-maraṇaṁ soka-parideva-dukkha-domanass'upāyāsā nirujjhanti
  \end{pali-hangtogether}
  \begin{pali-hangtogether}
    Evam'etassa kevalassa dukkha-kkhandhassa nirodho hoti
  \end{pali-hangtogether}
\end{pali-hang-continued}

\begin{english-verses}
  \begin{english-hang-firstline}
    But with the remainderless fading away and cessation of ignorance comes cessation of volitional formations;
  \end{english-hang-firstline}
  \begin{english-hangtogether-verses}
    With the cessation of volitional formations, cessation of consciousness;
  \end{english-hangtogether-verses}
  \begin{english-hangtogether-verses}
    With the cessation of consciousness, cessation of name-and-form;
  \end{english-hangtogether-verses}
  \begin{english-hangtogether-verses}
    With the cessation of name-and-form, cessation of the six sense bases;
  \end{english-hangtogether-verses}
  \begin{english-hangtogether-verses}
    With the cessation of the six sense bases, cessation of contact;
  \end{english-hangtogether-verses}
  \begin{english-hangtogether-verses}
    With the cessation of contact, cessation of feeling;
  \end{english-hangtogether-verses}
  \begin{english-hangtogether-verses}
    With the cessation of feeling, cessation of craving;
  \end{english-hangtogether-verses}
  \begin{english-hangtogether-verses}
    With the cessation of craving, cessation of clinging;
  \end{english-hangtogether-verses}
  \begin{english-hangtogether-verses}
    With the cessation of clinging, cessation of existence;
  \end{english-hangtogether-verses}
  \begin{english-hangtogether-verses}
    With the cessation of existence, cessation of birth;
  \end{english-hangtogether-verses}
  \begin{english-hangtogether-verses}
    With the cessation of birth, ageing-and-death, sorrow, lamentation, pain, displeasure, and despair cease.
  \end{english-hangtogether-verses}
  \begin{english-hangtogether-verses}
    Such is the cessation of this whole mass of suffering.
  \end{english-hangtogether-verses}
\end{english-verses}


\suttaRef{[SN 12.1]}



\sectionSubTitle{Just as Rivers}
\section{Paṭṭhāna-mātikā-pāṭho}
\label{patthana-matika-patho}

\begin{pali-hangtogether}
  \anglebracketleft\ \hspace{-0.5mm}Hetu-paccayo \hspace{-0.5mm}\anglebracketright\ ārammaṇa-paccayo adhipati-paccayo anantara-paccayo samanantara-paccayo saha-jāta-paccayo aññam'añña-paccayo nissaya-paccayo upanissaya-paccayo pure-jāta-paccayo pacchā-jāta-paccayo āsevana-paccayo kamma-paccayo vipāka-paccayo āhāra-paccayo indriya-paccayo jhāna-paccayo magga-paccayo sampayutta-paccayo vippayutta-paccayo atthi-paccayo n'atthi-paccayo vigata-paccayo avigata-paccayo
\end{pali-hangtogether}

\begin{english-hang-verses}
  Root condition, sense-object condition, predominant condition, immediate condition, directly immediate condition, coexistent condition, reciprocity condition, dependence condition, sufficing condition, pre-existent condition, post-existent condition, repetition condition, action condition, result condition, nutriment condition, faculty condition, jhāna condition, path condition, associated condition, separated condition, existence condition, non-existence condition, disappeared condition, non-dissappeared condition.
\end{english-hang-verses}

\suttaRef{[Dhs A]}



\sectionSubTitle{Just as Rivers}
\section{Adāsi-me ādi gāthā}
\label{adasi-me-adi-gatha}

\begin{pali-hangtogether}
  \anglebracketleft\ \hspace{-0.5mm}Adāsi me akāsi me \hspace{-0.5mm}\anglebracketright\ \\
  Ñāti-mittā sakhā ca me\\
  Petānaṁ dakkhiṇaṁ dajjā\\
  Pubbe katam'anussaraṁ
\end{pali-hangtogether}

\begin{english-verses}
  ``He gave to me [gifts], he did [things] for me.\\
  They're my relatives, friends and pals''.\\
  To the deceased one should give offerings,\\
  Remembering what was done before.
\end{english-verses}

\begin{pali-hang-continued}
  Na hi ruṇṇaṁ vā soko vā\\
  Yā v'aññā paridevanā\\
  Na taṁ petānam'atthāya\\
  Evaṁ tiṭṭhanti ñātayo
\end{pali-hang-continued}

\begin{english-verses}
  For neither weeping nor sorrow,\\
  Nor any form of lamentation\\
  Benefits the departed ones.\\
  Such is how the relatives remain.
\end{english-verses}

\begin{pali-hang-continued}
  Ayañ'ca kho dakkhinā dinnā\\
  Saṅghamhi supatiṭṭhitā\\
  Dīgharattaṁ hitāy'assa\\
  Ṭhānaso upakappati
\end{pali-hang-continued}

\begin{english-verses}
  And this offering that has been given\\
  And firmly established in the Saṅgha,\\
  Would be for their long-term welfare\\
  And arrives there immediately.
\end{english-verses}

\begin{pali-hang-continued}
  So ñāti-dhammo ca ayaṁ nidassito\\
  Petāna'pūjā ca katā uḷārā\\
  Balañ'ca bhikkhūnam'anuppadinnaṁ\\
  Tumhehi puññaṁ pasutaṁ anappakan'ti
\end{pali-hang-continued}

\begin{english-verses}
  And the duty of relatives has been shown,\\
  And lofty honouring of the departed done;\\
  Strength has also been given to the bhikkhus,\\
  And much merit accumulated by you all.
\end{english-verses}

\suttaRef{[Khp 7]}



\sectionSubTitle{Just as Rivers}
\section{Paṁsu-kūla for the Dead}
\label{pamsu-kula-dead}

\begin{pali-hangtogether}
  \anglebracketleft\ \hspace{-0.5mm}Aniccā vata saṅkhārā \hspace{-0.5mm}\anglebracketright\ \\
  Uppāda-vaya-dhammino\\
  Uppajjitvā nirujjhanti\\
  Tesaṁ vūpasamo sukho \hfill{[3x]}
\end{pali-hangtogether}

\begin{english-verses}
  Indeed, conditioned things cannot last\\
  \linkdest{endnote132-body}
  Their nature is to rise and cease;\makeatletter\hyperlink{endnote132-appendix}\Hy@raisedlink{{\pagenote{%
        \hypertarget{endnote132-appendix}{\hyperlink{endnote132-body}{WPN: ``Their nature is to rise and fall''}}}}}\makeatother\\
  Having arisen things must cease;\\
  Their stilling is true happiness.
\end{english-verses}

\suttaRef{[DN 16]}

% \vspace{-0.99em}

\begin{pali-hang-continued}
  \anglebracketleft\ \hspace{-0.5mm}Sabbe sattā \hspace{-0.5mm}\anglebracketright\ maranti ca\\
  Mariṁsu ca marissare\\
  Tath'ev'āhaṁ marissāmi\\
  N'atthi me ettha saṁsayo \hfill{[3x]}
\end{pali-hang-continued}

\begin{english-verses}
  All living beings are dying,\\
  Have died, and will die.\\
  In the same way, I will die,\\
  I have no doubt about this.
\end{english-verses}

\suttaRef{[Thai]}



\sectionSubTitle{Just as Rivers}
\section{Paṁsu-kūla for the Living}
\label{pamsu-kula-living}

\begin{pali-hangtogether}
  \anglebracketleft\ \hspace{-0.5mm}Aciraṁ vat'ayaṁ kāyo \hspace{-0.5mm}\anglebracketright\ \\
  Paṭhaviṁ adhisessati\\
  Chuḍḍho apeta-viññāṇo\\
  Niratthaṁ va kaliṅgaraṁ \hfill{[3x]}
\end{pali-hangtogether}

\begin{english-verses}
  All too soon, this body\\
  Will lie on the ground cast off,\\
  Bereft of consciousness,\\
  Like a useless scrap of wood.
\end{english-verses}

\suttaRef{[Dhp 41]}



\sectionSubTitle{Just as Rivers}
\section{Bhavatu-sabba-maṅgalaṁ}
\label{bhavatu-funeral}

\begin{pali-hangtogether}
  \anglebracketleft\ \hspace{-0.5mm}Bhavatu sabba-maṅgalaṁ \hspace{-0.5mm}\anglebracketright\ \\
  Rakkhantu sabba-devatā\\
  Sabba-buddh'ānubhāvena\\
  Sadā sotthī bhavantu te
\end{pali-hangtogether}

\begin{english-verses}
  May every blessing come to be.\\
  And all good spirits guard you well.\\
  Through the power of all Buddhas,\\
  May you always be at ease.
\end{english-verses}

\begin{pali-hang-continued}
  Bhavatu sabba-maṅgalaṁ\\
  Rakkhantu sabba-devatā\\
  Sabba-dhamm'ānubhāvena\\
  Sadā sotthī bhavantu te
\end{pali-hang-continued}

\begin{english-verses}
  May every blessing come to be.\\
  And all good spirits guard you well.\\
  Through the power of all Dhammas,\\
  May you always be at ease.
\end{english-verses}

\begin{pali-hang-continued}
  Bhavatu sabba-maṅgalaṁ\\
  Rakkhantu sabba-devatā\\
  Sabba-saṅgh'ānubhāvena\\
  Sadā sotthī \breathmark\ bhavantu te
\end{pali-hang-continued}

\begin{english-verses}
  May every blessing come to be.\\
  And all good spirits guard you well.\\
  Through the power of all Saṅghas,\\
  May you always be at ease.
\end{english-verses}

\suttaRef{[Trad]}

\bottomNav{recollection-of-impermanence}

\endgroup

