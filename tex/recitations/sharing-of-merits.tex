\ifdigitalversion
  \chapterOpeningPage{sharing-of-merits-compressed.jpg}
\else
  \chapterOpeningPage{sharing-of-merits.jpg}
\fi

\chapter{Sharing of Merits}

\begingroup
\setsechook{%
  % New page for each section.
  \clearpage%
  % Empty the default section number printing, so that we can handle it.
  \setsecnumformat{}%
}

\sectionSubTitle{Uddissan'ādhiṭṭhāna}
\setsecheadstyle{\sectionFmt}

\section{Sharing and Aspirations}
\label{uddissanadhitthana}

\begin{leader}
  \anglebracketleft\ \hspace{-0.5mm}Handa mayaṁ uddissan'ādhiṭṭhāna-gāthāyo \mbox{bhaṇāmase}~\hspace{-0.5mm}\anglebracketright\
\end{leader}

Iminā puñña-kammena upajjhāyā guṇ'uttarā\\
Ācariy'ūpakārā ca mātāpitā ca ñātakā\\
Suriyo candimā rājā guṇavantā narā'pi ca\\
Brahma-mārā ca indā ca lokapālā ca devatā\\
Yamo mittā manussā ca majjhattā verikā'pi ca\\
Sabbe sattā sukhī hontu puññāni pakatāni me\\
Sukhañ'ca tividhaṁ dentu khippaṁ pāpetha vo'mataṁ\\
Iminā puñña-kammena iminā uddissena ca\\
Khipp'āhaṁ sulabhe c'eva taṇh'ūpādāna-chedanaṁ\\
Ye santāne hīnā dhammā yāva nibbānato mamaṁ\\
Nassantu sabbadā yeva yattha jāto bhave bhave\\
Ujucittaṁ satipaññā sallekho viriyamhinā\\
Mārā labhantu nokāsaṁ kātuñ'ca viriyesu me\\
Buddh'ādhipavaro nātho dhammo nātho varuttamo\\
Nātho paccekabuddho ca saṅgho nāthottaro mamaṁ\\
Tes'ottam'ānubhāvena mār'okāsaṁ labhantu mā

\ifdigitalversion\bottomNav{closing-homage}\fi

\clearpage

\phantomsection

\begin{leader-english}
  \anglebracketleft\ \label{sharing-aspirations} Now let us recite the verses of sharing and aspiration \hspace{-0.5mm}\anglebracketright\
\end{leader-english}

\smallskip

\begin{english}
  Through the goodness that arises from my practice\\
  May my spiritual teachers and guides of great virtue\\
  My mother my father and my relatives\\
  \linkdest{endnote133-body}
  \ifninebythirteenversion\begin{english-hang-inline}\fi
  The sun and the moon \breathmark\ and all virtuous leaders of the world\makeatletter\hyperlink{endnote133-appendix}\Hy@raisedlink{{\pagenote{%
        \hypertarget{endnote133-appendix}{\hyperlink{endnote133-body}{While the celestial bodies themselves are not regarded as living beings, this passage refers to the similarly named young devas (\textit{candimā}/\textit{sūriyo devaputto}) residing there. See also SN 2.9-10. Furthermore, the term ``\textit{rājā}'' (leaders/kings) does not refer to ``\textit{guṅavantā}'' (virtuous people), but to ``\textit{suriyo candima}'' (sun and moon). A translation closer to the meaning of the Pāli would be: ``The sovereign Sun and Moon, and also virtuous people''.}}}}}\makeatother\\
  \ifninebythirteenversion\end{english-hang-inline}\fi
  May the highest gods and evil forces\\
  Celestial beings \breathmark\ guardian spirits of the earth\\
  And the Lord of Death\\
  May those who are friendly \breathmark\ indifferent or hostile\\
  May all beings receive the blessings of my life\\
  \linkdest{endnote134-body}
  \ifninebythirteenversion\begin{english-hang-inline}\fi
  May they soon attain the threefold bliss\makeatletter\hyperlink{endnote134-appendix}\Hy@raisedlink{{\pagenote{%
        \hypertarget{endnote134-appendix}{\hyperlink{endnote134-body}{This probably refers to  1. worldly or human happiness; 2. celestial or heavenly happiness; 3. transcendent happiness or \textit{Nibbāna}. Rhys Davids Pāli-English Dictionary - \textit{sukha}: ``Two kinds, viz \textit{kāyika} \& \textit{cetasika}; at Pts.i.188; several other pairs at AN.i.80; three (praise, wealth, heaven) Iti.67; another three (\textit{manussa}°, \textit{dibba}°, \textit{Nibbāna}°) Dhp-a.iii.51; four (possessing, making good use of possessions, having no debts, living a blameless life) AN.ii.69''}}}}}\makeatother
  \breathmark\ and realize the Deathless\\
  \ifninebythirteenversion\end{english-hang-inline}\fi
  Through the goodness that arises from my practice\\
  And through this act of sharing\\
  May all desires and attachments quickly cease\\
  And all harmful states of mind\\
  Until I realize Nibbāna\\
  In every kind of birth \breathmark\ may I have an upright mind\\
  With mindfulness and wisdom \breathmark\ austerity and vigor\\
  \linkdest{endnote135-body}
  \ifninebythirteenversion\begin{english-hang-inline}\fi
  May the forces of delusion\makeatletter\hyperlink{endnote135-appendix}\Hy@raisedlink{{\pagenote{%
        \hypertarget{endnote135-appendix}{\hyperlink{endnote135-body}{The \textit{Pāli} speaks about \textit{Māra's} forces.}}}}}\makeatother \thinspace
  not take hold \breathmark\ nor weaken my resolve\\
  \ifninebythirteenversion\end{english-hang-inline}\fi
  The Buddha is my excellent refuge\\
  Unsurpassed is the protection of the Dhamma\\
  \linkdest{endnote136-body}
  The Solitary Buddha is my noble guide\makeatletter\hyperlink{endnote136-appendix}\Hy@raisedlink{{\pagenote{%
        \hypertarget{endnote136-appendix}{\hyperlink{endnote136-body}{Even though a \textit{Paccekabuddha} does not (or is not able to) teach the path to \textit{Nibbāna}, he can nonetheless give guidance in good conduct and virtue; functioning as an inspirational role model.}}}}}\makeatother\\
  The Saṅgha is my supreme support\\
  Through the supreme power of all these\\
  \linkdest{endnote137-body}
  May darkness and delusion be dispelled\makeatletter\hyperlink{endnote137-appendix}\Hy@raisedlink{{\pagenote{%
        \hypertarget{endnote137-appendix}{\hyperlink{endnote137-body}{``Darkness and delusion'' is not a literal translation for \textit{Māra}.}}}}}\makeatother
\end{english}

\ifbfiveversion\vspace{-2pt}\fi

\suttaRef{[Trad]}

\ifdigitalversion\bottomNav{closing-homage}\fi

\sectionSubTitle{Sabba-patti-dāna}
\section{Sharing of All Merits}
\label{sharing-all-merits}

\begin{leader}
  \anglebracketleft\ \hspace{-0.5mm}Handa mayaṁ sabba-patti-dāna-gāthāyo bhaṇāmase \hspace{-0.5mm}\anglebracketright\
\end{leader}

Puññass'idāni katassa yān'aññāni katāni me\\
Tesañ'ca bhāgino hontu satt'ānant'āppamāṇakā

\begin{english-verses}
  May whatever living beings\\
  Without measure without end\\
  Partake of all the merit\\
  From the good deeds I have done
\end{english-verses}

Ye piyā guṇavantā ca mayhaṁ mātā-pitā-dayo\\
Diṭṭhā me c'āpy'adiṭṭhā vā aññe majjhatta-verino

\begin{english-verses}
  Those loved and full of goodness\\
  My mother and my father dear\\
  Beings seen by me and those unseen\\
  Those neutral and averse
\end{english-verses}

Sattā tiṭṭhanti lokasmiṁ te bhummā catu-yonikā\\
Pañc'eka-catu-vokārā saṁsarantā bhav'ābhave

\begin{english-verses}
  Beings established in the world\\
  From the three planes and four grounds of birth\\
  With five aggregates or one or four\\
  Wandering on from realm to realm
\end{english-verses}

Ñātaṁ ye patti-dānam'me anumodantu te sayaṁ\\
Ye c'imaṁ nappajānanti devā tesaṁ nivedayuṁ

\ifbfiveversion\clearpage\fi

\begin{english-verses}
  Those who know my act of dedication\\
  May they all rejoice in it\\
  And as for those yet unaware\\
  May the devas let them know
\end{english-verses}

Mayā dinnāna-puññānaṁ anumodana-hetunā\\
Sabbe sattā sadā hontu averā sukha-jīvino\\
Khema-ppadañ'ca pappontu tesāsā sijjhataṁ subhā

\begin{english-verses}
  By rejoicing in my sharing\\
  May all beings live at ease\\
  In freedom from hostility\\
  May their good wishes be fulfilled\\
  And may they all reach safety
\end{english-verses}

\suttaRef{[Thai]}

\ifdigitalversion\bottomNav{closing-homage}\fi

\sectionSubTitle{Peta-patti-dāna}
\section{Sharing of Merits with the Departed}
\label{sharing-merits-departed}

\begin{leader-only}
  \anglebracketleft\ \hspace{-0.5mm}Idaṁ me ñātinaṁ hotu \hspace{-0.5mm}\anglebracketright\
\end{leader-only}

\vspace{-0.99em}

\begin{pali-hang}
  Sukhitā hontu ñātayo\\
  Idaṁ no ñātinaṁ hotu sukhitā hontu ñātayo\\
  Idaṁ vo ñātinaṁ hotu sukhitā hontu ñātayo
\end{pali-hang}

\begin{english-verses}
  \begin{english-hang-first-line}
  May this be for my relatives \breathmark\ well and happy may the relatives be\\
  \end{english-hang-first-line}
  \begin{english-hang-together}
  May this be for our relatives \breathmark\ well and happy may the relatives be\\
  \end{english-hang-together}
  \begin{english-hang-together}
  May this be for your relatives \breathmark\ well and happy may the relatives be
  \end{english-hang-together}
\end{english-verses}

\suttaRef{[Pv 422]}

\ifdigitalversion\bottomNav{sharing-merits-devas}\fi

\sectionSubTitle{Devata-patti-dāna}
\section{Sharing of Merits with the Devas}
\label{sharing-merits-devas}

\begin{leader-only}
  \anglebracketleft\ \hspace{-0.5mm}Ettāvatā ca amhehi \hspace{-0.5mm}\anglebracketright\
\end{leader-only}

\vspace{-0.99em}

Sambhataṁ puñña-sampadaṁ\\
Sabbe devā anumodantu\\
Sabba-sampatti-siddhiyā

Ettāvatā ca amhehi\\
Sambhataṁ puñña-sampadaṁ\\
Sabbe bhūtā anumodantu\\
Sabba-sampatti-siddhiyā

Ettāvatā ca amhehi\\
Sambhataṁ puñña-sampadaṁ\\
Sabbe sattā anumodantu\\
Sabba-sampatti-siddhiyā

\ifdigitalversion\bottomNav{closing-homage}\fi

\ifafiveversion\clearpage\fi

\begin{english-verses}
  To the extent that all of us\\
  Have accumulated a wealth of merits;\\
  In this may all devas rejoice,\\
  For the attainment of all fortunes.
\end{english-verses}

\begin{english-verses}
  To the extent that all of us\\
  Have accumulated a wealth of merits;\\
\ifninebythirteenversion\clearpage\fi
  In this may all beings rejoice,\\
  For the attainment of all fortunes.
\end{english-verses}

\begin{english-verses}
  To the extent that all of us\\
  Have accumulated a wealth of merits;\\
  \ifbfiveversion\clearpage\fi
  In this may all creatures rejoice,\\
  For the attainment of all fortunes.
\end{english-verses}

\suttaRef{[MJG]}

\sectionSubTitle{Paramāya pūjāyañ'ca paṇidhiñ'ca}
\section{The Highest Honour and Aspirations}
\label{highest-honour-aspirations}

\begin{leader}
  \anglebracketleft\ \hspace{-0.5mm}Handa mayaṁ buddhapūjañ'ca paṇidhiñ'ca karomase \hspace{-0.5mm}\anglebracketright\
\end{leader}

Buddhaṁ jīvita-pariyantaṁ saraṇaṁ gacchāmi

\begin{english}
  Until life ends I go to the Buddha for refuge
\end{english}

Dhammaṁ jīvita-pariyantaṁ saraṇaṁ gacchāmi

\begin{english}
  Until life ends I go to the Dhamma for refuge
\end{english}

Saṅghaṁ jīvita-pariyantaṁ saraṇaṁ gacchāmi

\begin{english}
  Until life ends I go to the Saṅgha for refuge
\end{english}

Iminā puñña-kammena

\begin{english}
  By this meritorious action
\end{english}

Mā me bāla-samāgamo

\begin{english}
  May I not associate with fools
\end{english}

Sataṁ samāgamo hotu

\begin{english}
  With the wise may I associate
\end{english}

Yāva nibbānapattiyā

\begin{english}
  Until the attainment of Nibbāna
\end{english}

\suttaRef{[Sri Lanka]}

Yo kho bhikkhu vā bhikkhunī vā upāsako vā upāsikā vā

\begin{english}
  Any bhikkhu \breathmark\ bhikkhunī \breathmark\ male or female lay follower
\end{english}


Dhamm'ānudhamma-paṭipanno viharati

\begin{english}
  Who dwells practicing according to the Dhamma
\end{english}

\ifbfiveversion\clearpage\fi

Sāmīci-paṭipanno

\begin{english}
  Practicing properly
\end{english}

Anudhammacārī

\begin{english}
  Behaving according to the Dhamma
\end{english}

\begin{pali-hang}
So tathāgataṁ sakkaroti garuṁ karoti māneti pūjeti apaciyati
\end{pali-hang}

\begin{english-hang}
  Respects \breathmark\ esteems \breathmark\ cherishes \breathmark\ honours and pays homage to the Tathāgata
\end{english-hang}

Paramāya pūjāya

\begin{english}
  With the highest honour
\end{english}

\suttaRef{[DN 16]}

Tasmā

\begin{english}
  Therefore
\end{english}

Imāya dhamm'ānudhamma-paṭipattiyā buddhaṁ pūjemi\\
Paramāya pūjāya

\begin{english}
  By this Dhamma practice according to the Dhamma\\
  I honour the Buddha with the highest honour
\end{english}

\begin{pali-hang}
  Addhā imāya paṭipadāya jāti-jarā-byādhi-maraṇamhā parimuccissāmi
\end{pali-hang}

\begin{english}
  Surely by this way of practice\\
  I will be free from birth \breathmark\ ageing \breathmark\ sickness and death
\end{english}

Idaṁ me puññaṁ āsavakkhayā-vahaṁ hotu

\begin{english}
  May my merit lead to the destruction of the taints
\end{english}

Idaṁ me puññaṁ nibbānassa paccayo hotu

\begin{english}
  May my merit be a condition for the attainment of Nibbāna
\end{english}

\suttaRef{[Sri Lanka]}

\ifdigitalversion\bottomNav{closing-homage}\fi

\endgroup
