\documentclass{article}
\usepackage{xcolor}
\usepackage{amsmath} % For unicode arrows
\usepackage{tcolorbox} % For better formatting and highlighting

% Define a custom command for change tracking
\newcommand{\change}[3]{
    \begin{tcolorbox}[colback=gray!10, colframe=gray!80, title=Dedication of Offerings]
        \textbf{WPN:} \textit{#1} \quad\(\Rightarrow\)\quad \textbf{SBS:} \textit{#2} \\
        \textbf{Reasoning:} \\
        #3
    \end{tcolorbox}
}

\begin{document}

\change{Lord}
       {Worthy One}
       {The underlying Pāli term is ``\textit{Arahant}''. ``Lord'', however, has connotations that do not fit well to the way the Buddha is portrayed in the discourses. In dictionaries ``lord'' is commonly defined as: ``an appellation for a person or deity who has authority, control, or power over others, acting like a master, a chief, or a ruler''. The ``Worthy One'' seems a better choice of terms, since it is also how ``\textit{Arahant}'' was used in pre-Buddhist era. PTS explains: ``[Vedic \textit{arhant}, ppr. of \textit{arhati} (see \textit{arahati}), meaning deserving, worthy]''. Throughout this chanting book, all occurrences of ``\textit{Arahant}'' have therefore been consistently translated as ``Worthy One'', thus replacing previous translations, such as ``The Lord'', ``Noble One'' etc.}

\end{document}

Homage to the Buddha
\makeatletter\hyperlink{endnote3-appendix}\Hy@raisedlink{{\pagenote{%
        \hypertarget{endnote3-appendix}{\hyperlink{endnote3-body}{WPN: ``He trains perfectly those who wish to be trained''. The aspect of wishing to be trained is not found in the \textit{Pāli}.}}}}}\makeatother

Homage to the Buddha
\makeatletter\hyperlink{endnote4-appendix}\Hy@raisedlink{{\pagenote{%
        \hypertarget{endnote4-appendix}{\hyperlink{endnote4-body}{WPN: ``He has explained the spiritual life of complete purity''. While ``spiritual life'' is not a bad translation, for the sake of consistency with the rest of the chanting book, this occurrence was changed to ``holy life''.}}}}}\makeatother

Homage to the Sangha
\makeatletter\hyperlink{endnote6-appendix}\Hy@raisedlink{{\pagenote{%
        \hypertarget{endnote6-appendix}{\hyperlink{endnote6-body}{WPN: ``Who have practiced insightfully''}}}}}\makeatother

Homage to the Sangha
\makeatletter\hyperlink{endnote7-appendix}\Hy@raisedlink{{\pagenote{%
        \hypertarget{endnote7-appendix}{\hyperlink{endnote7-body}{WPN: ``Those who practice with integrity''}}}}}\makeatother

Homage to the Sangha
\makeatletter\hyperlink{endnote149-appendix}\Hy@raisedlink{{\pagenote{%
        \hypertarget{endnote149-appendix}{\hyperlink{endnote149-body}{WPN: ``They give occasion for incomparable goodness to arise in the world''}}}}}\makeatother

Salutation to the Triple Gem
\makeatletter\hyperlink{endnote8-appendix}\Hy@raisedlink{{\pagenote{%
        \hypertarget{endnote8-appendix}{\hyperlink{endnote8-body}{WPN: ``The teaching of the Lord like a lamp''}}}}}\makeatother

Salutation to the Triple Gem
\makeatletter\hyperlink{endnote9-appendix}\Hy@raisedlink{{\pagenote{%
        \hypertarget{endnote9-appendix}{\hyperlink{endnote9-body}{WPN: ``Illuminating the path and its fruit, the Deathless''}}}}}\makeatother

Salutation to the Triple Gem
\makeatletter\hyperlink{endnote10-appendix}\Hy@raisedlink{{\pagenote{%
        \hypertarget{endnote10-appendix}{\hyperlink{endnote10-body}{WPN: ``That which is beyond the conditioned world''}}}}}\makeatother

Salutation to the Triple Gem
\makeatletter\hyperlink{endnote11-appendix}\Hy@raisedlink{{\pagenote{%
        \hypertarget{endnote11-appendix}{\hyperlink{endnote11-body}{WPN: ``To that which is worthy''. This passage refers to the triple (\textit{taya}) gems and not just to the \textit{Saṅgha}.}}}}}\makeatother

Salutation to the Triple Gem
\makeatletter\hyperlink{endnote14-appendix}\Hy@raisedlink{{\pagenote{%
        \hypertarget{endnote14-appendix}{\hyperlink{endnote14-body}{WPN: ``grief''}}}}}\makeatother

Salutation to the Triple Gem
\makeatletter\hyperlink{endnote15-appendix}\Hy@raisedlink{{\pagenote{%
        \hypertarget{endnote15-appendix}{\hyperlink{endnote15-body}{WPN: ``In brief the five focuses of identity are dukkha''}}}}}\makeatother

Salutation to the Triple Gem
\makeatletter\hyperlink{endnote16-appendix}\Hy@raisedlink{{\pagenote{%
        \hypertarget{endnote16-appendix}{\hyperlink{endnote16-body}{WPN: ``Attachment to mental formations''. While the Pāli term ``saṅkhārā'' only means ``formations'', Bhikkhu Bodhi’s rendering as ``volitional formations'' captures well the intentional/volitional forces behind the formative nature of the mind.}}}}}\makeatother

Salutation to the Triple Gem
\makeatletter\hyperlink{endnote17-appendix}\Hy@raisedlink{{\pagenote{%
        \hypertarget{endnote17-appendix}{\hyperlink{endnote17-body}{WPN: ``Attachment to sense-consciousness''}}}}}\makeatother

Salutation to the Triple Gem
\makeatletter\hyperlink{endnote18-appendix}\Hy@raisedlink{{\pagenote{%
        \hypertarget{endnote18-appendix}{\hyperlink{endnote18-body}{WPN: ``Mental formations are impermanent''}}}}}\makeatother

Salutation to the Triple Gem
\makeatletter\hyperlink{endnote19-appendix}\Hy@raisedlink{{\pagenote{%
        \hypertarget{endnote19-appendix}{\hyperlink{endnote19-body}{WPN: ``Sense-consciousness is impermanent''}}}}}\makeatother

Salutation to the Triple Gem
\makeatletter\hyperlink{endnote20-appendix}\Hy@raisedlink{{\pagenote{%
        \hypertarget{endnote20-appendix}{\hyperlink{endnote20-body}{WPN: ``Mental formations are not-self''}}}}}\makeatother

Salutation to the Triple Gem
\thinspace\makeatletter\hyperlink{endnote21-appendix}\Hy@raisedlink{{\pagenote{%
        \hypertarget{endnote21-appendix}{\hyperlink{endnote21-body}{WPN: ``Sense-consciousness is not-self''}}}}}\makeatother

Salutation to the Triple Gem
\makeatletter\hyperlink{endnote22-appendix}\Hy@raisedlink{{\pagenote{%
        \hypertarget{endnote22-appendix}{\hyperlink{endnote22-body}{WPN: ``All conditions are transient''}}}}}\makeatother

Salutation to the Triple Gem
\thinspace\makeatletter\hyperlink{endnote23-appendix}\Hy@raisedlink{{\pagenote{%
        \hypertarget{endnote23-appendix}{\hyperlink{endnote23-body}{WPN: ``There is no self in the created or the uncreated''. While this is not a very accurate translation, it is indeed the case that the term ``\textit{sabbe dhammā}'' includes the uncreated, \textit{Nibbāna} (see AN 5.32).}}}}}\makeatother

Salutation to the Triple Gem
\makeatletter\hyperlink{endnote24-appendix}\Hy@raisedlink{{\pagenote{%
        \hypertarget{endnote24-appendix}{\hyperlink{endnote24-body}{WPN: ``All of us are bound by birth ageing and death''}}}}}\makeatother

Salutation to the Triple Gem
\makeatletter\hyperlink{endnote25-appendix}\Hy@raisedlink{{\pagenote{%
        \hypertarget{endnote25-appendix}{\hyperlink{endnote25-body}{WPN: ``grief''}}}}}\makeatother

Salutation to the Triple Gem
\makeatletter\hyperlink{endnote27-appendix}\Hy@raisedlink{{\pagenote{%
        \hypertarget{endnote27-appendix}{\hyperlink{endnote27-body}{WPN: ``All of us are bound by birth ageing and death''}}}}}\makeatother

Salutation to the Triple Gem
\makeatletter\hyperlink{endnote28-appendix}\Hy@raisedlink{{\pagenote{%
        \hypertarget{endnote28-appendix}{\hyperlink{endnote28-body}{WPN: ``Being fully equipped with the bhikkhus'system of training''}}}}}\makeatother

Respect for the Dhamma
\makeatletter\hyperlink{endnote29-appendix}\Hy@raisedlink{{\pagenote{%
        \hypertarget{endnote29-appendix}{\hyperlink{endnote29-body}{WPN: ``what not'': 'What not' is usually followed by what is similar. Furthermore, the meaning of \textit{adhamma} in this context describes not only the lack of \textit{Dhamma}, but the practice of wrong \textit{Dhamma}.}}}}}\makeatother

Respect for the Dhamma
\makeatletter\hyperlink{endnote30-appendix}\Hy@raisedlink{{\pagenote{%
        \hypertarget{endnote30-appendix}{\hyperlink{endnote30-body}{WPN: ``lack of \textit{Dhamma}'' This translation is problematic, because a mere ``lack of \textit{Dhamma}'' does not lead to rebirth in hell; otherwise all non-Buddhists would be destined to hell. In reality, it is the view and practice of ``wrong \textit{Dhamma}'' that leads to hell, which is also substantiated by the Commentary, which defines ``\textit{adhamma}'' as the opposite (\textit{paṭipakkha}) of true \textit{Dhamma}.}}}}}\makeatother

Going to True and False Refuges
\makeatletter\hyperlink{endnote32-appendix}\Hy@raisedlink{{\pagenote{%
        \hypertarget{endnote32-appendix}{\hyperlink{endnote32-body}{WPN: ``from suffering''}}}}}\makeatother

Going to True and False Refuges
\makeatletter\hyperlink{endnote162-appendix}\Hy@raisedlink{{\pagenote{%
        \hypertarget{endnote162-appendix}{\hyperlink{endnote162-body}{WPN: ``Whoever goes to refuge''}}}}}\makeatother

Going to True and False Refuges
\makeatletter\hyperlink{endnote163-appendix}\Hy@raisedlink{{\pagenote{%
        \hypertarget{endnote163-appendix}{\hyperlink{endnote163-body}{WPN: ``In the Triple Gem''. It has been shown in Āyasma Aggacitta's ``Do We Go to the Triple Gem for Refuge?'' that one does not go for refuge in the triple gem (the Buddha, Dhamma, and Ariya-saṅgha), but the discourses repeatedly depict the Buddha, Dhamma, and Bhikkhu-saṅgha as the community to take refuge in (DN 2, MN 91, SN 12.46, AN 2.16 etc.).}}}}}\makeatother

The Three Characteristics
\makeatletter\hyperlink{endnote34-appendix}\Hy@raisedlink{{\pagenote{%
        \hypertarget{endnote34-appendix}{\hyperlink{endnote34-body}{WPN: ``Impermanent are all conditioned things''}}}}}\makeatother

\makeatletter\hyperlink{endnote36-appendix}\Hy@raisedlink{{\pagenote{%
        \hypertarget{endnote36-appendix}{\hyperlink{endnote36-body}{WPN: ``Dukkha are all conditioned things''}}}}}\makeatother

The Three Characteristics
\makeatletter\hyperlink{endnote37-appendix}\Hy@raisedlink{{\pagenote{%
        \hypertarget{endnote37-appendix}{\hyperlink{endnote37-body}{WPN: ``From the floods dry land they reach''}}}}}\makeatother

The Three Characteristics
\makeatletter\hyperlink{endnote38-appendix}\Hy@raisedlink{{\pagenote{%
        \hypertarget{endnote38-appendix}{\hyperlink{endnote38-body}{WPN: ``Living withdrawn so hard to do''}}}}}\makeatother

The Burdens
\makeatletter\hyperlink{endnote39-appendix}\Hy@raisedlink{{\pagenote{%
        \hypertarget{endnote39-appendix}{\hyperlink{endnote39-body}{WPN: ``The beast of burden though is man''. The Pāli word ``\textit{puggalo}'' is in masculine, which is the expected grammatical form even if a term refers to males and females alike, as is probably the case here. Furthermore, the phrase ``beast of burden'' is an English idiomatic expression, signifying ``an animal used for heavy work such as carrying or pulling things'' (Oxford Dictionary).}}}}}\makeatother

On Protection
\makeatletter\hyperlink{endnote41-appendix}\Hy@raisedlink{{\pagenote{%
        \hypertarget{endnote41-appendix}{\hyperlink{endnote41-body}{WPN: ``The dust of passions all the more''. The Pāli only speaks of stirring up dust, but the Commentary explains that it refers to the dust of \textit{kilesā}. As a translation for \textit{kilesā}, the term ``defilements'' has a broader scope than just ``passions'', wherefore the former has been given preference.}}}}}\makeatother

A Single Excellent Night
\makeatletter\hyperlink{endnote156-appendix}\Hy@raisedlink{{\pagenote{%
        \hypertarget{endnote156-appendix}{\hyperlink{endnote156-body}{WPN: ``\textit{viddhā}''. The Siam edition has \textit{viddhā} (pierced) whereas the Burmese and Sri Lankan edition of the same text have \textit{vidvā} (the wise), which seems to be the variant that better conveys the intended meaning of the text.}}}}}

A Single Excellent Night
\makeatletter\hyperlink{endnote43-appendix}\Hy@raisedlink{{\pagenote{%
        \hypertarget{endnote43-appendix}{\hyperlink{endnote43-body}{WPN: ``Such insight is one's strength''}}}}}\makeatother

A Single Excellent Night
\makeatletter\hyperlink{endnote44-appendix}\Hy@raisedlink{{\pagenote{%
        \hypertarget{endnote44-appendix}{\hyperlink{endnote44-body}{WPN: ``Thus for a night of non-decline''}}}}}\makeatother

A Single Excellent Night
\makeatletter\hyperlink{endnote45-appendix}\Hy@raisedlink{{\pagenote{%
        \hypertarget{endnote45-appendix}{\hyperlink{endnote45-body}{WPN: ``a shining night of prosperty''}}}}}\makeatother

The Highest Blessings
\makeatletter\hyperlink{endnote47-appendix}\Hy@raisedlink{{\pagenote{%
        \hypertarget{endnote47-appendix}{\hyperlink{endnote47-body}{WPN: ``Was staying at \textit{Sāvatthī}''}}}}}\makeatother

The Highest Blessings
\makeatletter\hyperlink{endnote155-appendix}\Hy@raisedlink{{\pagenote{%
        \hypertarget{endnote155-appendix}{\hyperlink{endnote155-body}{WPN: ``that harm no being''}}}}}

The Highest Blessings
\makeatletter\hyperlink{endnote151-appendix}\Hy@raisedlink{{\pagenote{%
        \hypertarget{endnote151-appendix}{\hyperlink{endnote151-body}{WPN: ``And hearing the Dhamma frequently taught''}}}}}

The Highest Blessings
\makeatletter\hyperlink{endnote151-appendix}\Hy@raisedlink{{\pagenote{%
        \hypertarget{endnote151-appendix}{\hyperlink{endnote151-body}{WPN: ``And sharing often the words of Dhamma''}}}}}

The Highest Blessings
\makeatletter\hyperlink{endnote48-appendix}\Hy@raisedlink{{\pagenote{%
        \hypertarget{endnote48-appendix}{\hyperlink{endnote48-body}{WPN: ``ardent committed''}}}}}\makeatother \thinspace

The Buddha's Words on Loving-Kindness
\makeatletter\hyperlink{endnote49-appendix}\Hy@raisedlink{{\pagenote{%
        \hypertarget{endnote49-appendix}{\hyperlink{endnote49-body}{WPN: ``Even''}}}}}\makeatother

The Buddha's Words on Loving-Kindness
\makeatletter\hyperlink{endnote50-appendix}\Hy@raisedlink{{\pagenote{%
        \hypertarget{endnote50-appendix}{\hyperlink{endnote50-body}{WPN: ``to fixed views''. This paragraph deals with \textit{Anāgāmis}, which becomes apparent from the closing statement in which it is said that by this practice one becomes free from sense desires and is not born again into this world (sense sphere). The limitation of loving-kindness practice leading ``only'' up to \textit{Anāgāmihood} also finds confirmation by AN 4.126. Now, since an \textit{Anāgāmi} has right view, which is the first factor of the noble eightfold path (even \textit{Arahants} hold right view; AN 10.112), it would therefore not be correct to say that he holds no views at all. Furthermore, even an \textit{Anāgāmi} may still have some minor grasping to (right) view, as there can still be moments of \textit{māna} (identification/conceit), which is overcome only by the \textit{Arahant}. I therefore conclude that this passage here refers specifically to ``wrong views'' and does not include ``right view'', since wrong views are the only types of views that an \textit{Anāgāmi} has entirely left behind.}}}}}\makeatother

Setting in Motion the Wheel of Dhamma
\makeatletter\hyperlink{endnote51-appendix}\Hy@raisedlink{{\pagenote{%
        \hypertarget{endnote51-appendix}{\hyperlink{endnote51-body}{WPN: ``It is this Noble Eightfold Path''}}}}}\makeatother

Setting in Motion the Wheel of Dhamma
\makeatletter\hyperlink{endnote53-appendix}\Hy@raisedlink{{\pagenote{%
        \hypertarget{endnote53-appendix}{\hyperlink{endnote53-body}{WPN: ``grief''}}}}}\makeatother

Setting in Motion the Wheel of Dhamma
\makeatletter\hyperlink{endnote54-appendix}\Hy@raisedlink{{\pagenote{%
        \hypertarget{endnote54-appendix}{\hyperlink{endnote54-body}{WPN: ``In brief the five focuses of identity are \textit{dukkha}''}}}}}\makeatother

Setting in Motion the Wheel of Dhamma
\makeatletter\hyperlink{endnote55-appendix}\Hy@raisedlink{{\pagenote{%
        \hypertarget{endnote55-appendix}{\hyperlink{endnote55-body}{WPN: ``Light arose''}}}}}\makeatother

Setting in Motion the Wheel of Dhamma
\makeatletter\hyperlink{endnote56-appendix}\Hy@raisedlink{{\pagenote{%
        \hypertarget{endnote56-appendix}{\hyperlink{endnote56-body}{WPN: ``Now this Noble Truth of \textit{dukkha}''}}}}}\makeatother

Setting in Motion the Wheel of Dhamma
\makeatletter\hyperlink{endnote158-appendix}\Hy@raisedlink{{\pagenote{%
        \hypertarget{endnote158-appendix}{\hyperlink{endnote158-body}{WPN: ``As long \textit{bhikkhus} as''}}}}}\makeatother \thinspace

Setting in Motion the Wheel of Dhamma
\makeatletter\hyperlink{endnote159-appendix}\Hy@raisedlink{{\pagenote{%
        \hypertarget{endnote159-appendix}{\hyperlink{endnote159-body}{WPN: ``But when \textit{bhikkhus}''}}}}}\makeatother \thinspace

The Noble Eightfold Path
\makeatletter\hyperlink{endnote61-appendix}\Hy@raisedlink{{\pagenote{%
        \hypertarget{endnote61-appendix}{\hyperlink{endnote61-body}{WPN: ``covetousness''}}}}}\makeatother\thinspace

The Noble Eightfold Path
\makeatletter\hyperlink{endnote62-appendix}\Hy@raisedlink{{\pagenote{%
        \hypertarget{endnote62-appendix}{\hyperlink{endnote62-body}{WPN: ``covetousness and grief''}}}}}\makeatother

The Noble Eightfold Path
\makeatletter\hyperlink{endnote63-appendix}\Hy@raisedlink{{\pagenote{%
        \hypertarget{endnote63-appendix}{\hyperlink{endnote63-body}{WPN: ``He abides contemplating mind-objects as mind-objects''. Since ``mind-object'' is not an ideal translation for ``\textit{dhamma}'' in this context, it is preferable to leave ``\textit{dhamma}'' untranslated here.}}}}}\makeatother
